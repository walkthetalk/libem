%
% author:	Ni Qingliang
% date:		2012-11-07
%
\startcomponent cpn-svc-model
\product wm-sw-arch

\chapter{业务模型}

业务模型有两方面内容,一是如何驱动业务运转,二是业务间如何通信。

\section{事件驱动}

在我们的软件中,以【事件】来驱动程序运转,
事件驱动的好处就是在有外部事件输入的情况下才需要处理,空闲状态下什么都不需要做。
也就是说没有事件输入的情况下,程序处于暂停状态,不占用 CPU。
这样做的好处是显而易见的,避免程序空转消耗 CPU。
这个事件驱动模型中还用到了 Linux 的 epoll 机制,
同时它的可行性还得益于 Linux 中视一切为文件的思想
(比如, timer、 signal、 message queue、 socket、 pipe 等等在 Linux 中都是文件,
当然设备文件也是文件)。

其基本流程为:
\startigNum
\item 一个进程(线程)会等在一系列文件上;
\item 其中任何一个文件有变化,比如 socket 收包, timer 超时,
都会以事件的形式唤醒此进程继续执行;
\item 而此进程则会根据事件源(哪个文件)、事件类型来确定如何处理;
\item 处理完毕后则进行下一次等待。
\stopigNum

一般而言一个进程(线程)可能有多种状态,不同状态下接受(可以处理)不同的事件,
这样就需要用到状态机,对于熔接机而言,为简化问题,暂时不引入状态机,进程只有一个基本状态:等待。
处理事件是一个过程,不是状态,一般处理完事件就会改变程序的状态,
但由于我们的程序只有一个状态,因此处理完事件后状态不变。
如果日后发现确实需要状态机,可以引入 C++ 的 boost 库,其中有两套状态机,择一而用。

\placefigure[here][fig:wm_evt_drv]
{事件驱动}
{\externalfigure[mp/umls-1.pdf]}

\reffig{wm_evt_drv} 中列出了业务执行模型中的主要 class。
其中 \ccmm{poller} 相当于轮询器,负责等待新事件的到来。
而 \ccmm{pollee} 为接受轮询者,为接口类,定义了一个接口 \ccmm{dispose}。
\ccmm{poller} 所提供的入口函数 \ccmm{run} 中会等待新事件的到来,
然后调用 \ccmm{pollee} 的 \ccmm{dispose} 接口来处理事件。
 \ccmm{serveree}、 \ccmm{connectee}、 \ccmm{clientee}、 \ccmm{timeree} 等均继承自
接口 \ccmm{pollee},并实现了 \ccmm{pollee} 所定义的接口 \ccmm{dispose}。
其中前三个都与 socket 有关,目前以 TCP 实现。
后面视具体需求逐步添加对 message queue、 signal、 device file 等的支持。

\section{消息通信}

既然有多进程、多线程的存在,就必须为其制定通信方式,上一节中已经介绍了几种 \ccmm{pollee},
可以优先选用 socket 方式以避免进程、线程的选择所带来的影响。
通信机制确立以后,还有一个问题就是消息格式,或者说数据序列化、反序列化的方式。
目前应用较广的是 msgpack 和 protobuf,通过对比测试,基于性能和复杂性方面的考虑,
最终选用 msgpack 作为数据序列化、反序列化的工具,
为了简化其使用,对其进行了进一步封装,主要有三个(模板)类:
\startigNum
\item \ccmm{msg<mid_t id>},即消息,其模板参数为消息 ID (枚举类型),
我们需要为每一个消息 ID 对此模板进行特化;
\startitem
\ccmm{recver},即消息接收器,它提供了几个接口,使用示例:
\startCPP
msg::recver pac;
pac.fill_from(obj);			// 1. 由 obj 接收消息
switch (pac.id()) {			// 2. 取得消息 ID
	case msg::mid_t::test1: {
		msg::msg<msg::mid_t::test1> msg;
		pac.convert(msg);	// 3. 消息的反序列化
		...
\stopCPP
\stopitem
\startitem
\ccmm{sender},即消息发送器,示例如下:
\startCPP
msg::sender pac;
msg::msg<msg::mid_t::test1> msg;
...				// 1. 填充消息
pac.fill_to(msg, obj);		// 2. 将消息序列化并由 obj 发送消息
\stopCPP
\stopitem
\stopigNum
其中 \ccmm{fill_from} 和 \ccmm{fill_to} 是两个模板成员函数,
会分别调用各自参数 \ccmm{obj} 的 \ccmm{recv} 和 \ccmm{send} 成员函数来接收和发送消息,
你也可以使用所重载的另外一个版本来自己指定 \ccmm{obj} 的成员函数。
\startnotepar
\ccmm{fill_to} 会将消息 ID 和消息内容全部打包。
\stopnotepar

\subsection{定义消息}
示例代码如下:
\startCPP
template<>
class msg<mid_t::test1> {
public:
	msg() : a(0), b(0) {}
public:
	uint8_t a;
	uint32_t b;
public:
	MSGPACK_DEFINE(a, b)
};
\stopCPP
其中 \ccmm{MSGPACK_DEFINE} 是一个宏,用来定义要(反)序列化哪些成员变量及其(反)序列化的顺序。

%\color[red]{\bfa\textbackslash\textbackslash TODO: 请添加相关内容。}

\stopcomponent

