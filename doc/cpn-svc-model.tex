%
% author:	Ni Qingliang
% date:		2012-11-07
%
\startcomponent cpn-svc-model
\product wm-sw-arch

\chapter{业务模型}

\section{业务模型的要素}
业务模型有三个要素:
\startigNum
\item 如何驱动业务运转;
\item 业务间如何通信;
\item 业务如何运转。
\stopigNum

\subsection[sec:evt_drv]{事件驱动}

在我们的软件中,以【事件】来驱动程序运转,
事件驱动的好处就是在有外部事件输入的情况下才需要处理,空闲状态下什么都不需要做。
也就是说没有事件输入的情况下,程序处于暂停状态(对应于 Linux 中的  SLEEP)。
这样做的好处是显而易见的,避免程序空转消耗 CPU。
这个事件驱动模型中还用到了 Linux 的 epoll 机制,
同时它的可行性还得益于 Linux 中视一切为文件的思想
(比如, timer、 signal、 message queue、 socket、 pipe 等等在 Linux 中都是文件,
当然设备文件也是文件)。

其基本流程为:
\startigNum
\item[item:wait] 一个进程(线程)会执行系统调用 \ccmm{epoll_wait} 等在一系列文件上;
\item 其中任何一个文件有变化,比如 socket 收包, timer 超时,
都会以事件的形式唤醒此进程(线程)继续执行;
\item 而此进程则会根据事件源(哪个文件)、事件类型调用用户所注册的回调函数进行处理;
\item 处理完毕后则进行下一次等待,同\refitem{wait}。
\stopigNum

一般而言一项业务可能有多种状态(不同于 OS 中进程状态),不同状态下接受(可以处理)不同的事件,
这样就需要用到状态机,关于状态机参见\refsec{state_machine}。

\placefigure[here][fig:wm_evt_drv]
{事件驱动}
{\externalfigure[mp/umls-1.pdf]}

\reffig{wm_evt_drv} 中列出了业务执行模型中的主要 class。
其中 \ccmm{poller} 相当于轮询器,负责等待新事件的到来。
而 \ccmm{pollee} 为接受轮询者,为接口类,定义了一个接口 \ccmm{dispose}。
\ccmm{poller} 所提供的入口函数 \ccmm{run} 中会等待新事件的到来,
然后调用 \ccmm{pollee} 的 \ccmm{dispose} 接口来处理事件。
 \ccmm{serveree}、 \ccmm{connectee}、 \ccmm{clientee}、 \ccmm{timeree} 等均继承自
接口 \ccmm{pollee},并实现了 \ccmm{pollee} 所定义的接口 \ccmm{dispose}。
其中前三个都与 socket 有关,目前以 TCP 实现。
后面视具体需求逐步添加对 message queue、 signal、 device file 等的支持。

\subsection{消息通信}

既然有多进程、多线程的存在,就必须为其制定通信方式,上一节中已经介绍了几种 \ccmm{pollee},
可以优先选用 socket 方式以避免进程、线程的选择所带来的影响。
通信机制确立以后,还有一个问题就是消息格式,或者说数据序列化、反序列化的方式。
目前应用较广的是 msgpack 和 protobuf 两种 SDK,通过对比测试,基于性能和复杂性方面的考虑,
最终选用 msgpack 作为数据序列化、反序列化的工具,
为了简化其使用,对其进行了进一步封装,主要有三个(模板)类:
\startigNum
\item \ccmm{msg<mid_t id>},即消息,其模板参数为消息 ID (枚举类型),
我们需要为每一个消息 ID 对此模板进行特化;
\startitem
\ccmm{recver},即消息接收器,它提供了几个接口,使用示例\footnote{%
\useURL[wiki-table-driven][http://en.wikipedia.org/wiki/Table-driven]
\ccmm{switch / case} 仅为示例,如果消息比较多,则改用表 Table-Driven
\from[wiki-table-driven] 的方式来实现。}:
\startCPP
msg::recver pac;
pac.fill_from(obj);		// 1. 由 obj 接收消息
switch (pac.id()) {		// 2. 取得消息 ID
case msg::mid_t::test1: {
	msg::msg<msg::mid_t::test1> msg;
	pac.convert(msg);	// 3. 消息的反序列化
	...			// 4. 处理消息
	break;
\stopCPP
\stopitem
\startitem
\ccmm{sender},即消息发送器,示例如下\footnote{%
\ccmm{fill_to} 会将消息 ID 和消息内容全部打包。}:
\startCPP
msg::sender pac;
msg::msg<msg::mid_t::test1> msg;
...				// 1. 填充消息
pac.fill_to(msg, obj);		// 2. 将消息序列化并由 obj 发送消息
\stopCPP
\stopitem
\stopigNum
其中 \ccmm{fill_from} 和 \ccmm{fill_to} 是两个模板成员函数,
会分别调用各自参数 \ccmm{obj} 的 \ccmm{recv} 和 \ccmm{send} 成员函数来接收和发送消息,
你也可以使用所重载的另外一个版本来自己指定 \ccmm{obj} 的成员函数。

\subsubsection{定义消息}
示例代码如下:
\startCPP
template<>
class msg<mid_t::test1> {
public:
	msg() : a(0), b(0) {}
public:
	uint8_t a;
	uint32_t b;
public:
	MSGPACK_DEFINE(a, b)
};
\stopCPP
其中 \ccmm{MSGPACK_DEFINE} 是一个宏,用来定义要(反)序列化哪些成员变量及其(反)序列化的顺序。

\subsection[sec:state_machine]{状态机}
每一项业务都会有自己的流程,而流程中又会有很多状态,用状态机描述流程具有简单、直观、易维护等特点。
而涉及到状态机的实现,我们既可以用一些比较成熟的库,也可以手工编写(handcrafted)。
这两种方式都能达到目的,但是前者的健壮性、可扩展性、可维护性无疑更好一些,
当然可能会有一些性能上的开销。
目前 C++ 语言的状态机库用的较多的为 boost 中的 MSM (meta state machine) 和 statechart,
其中 FSM 可以认为是一个简化版的 statechart,当前这两个库还有其他区别,比如对 UML 的支持。
我们最终选用的是 statechart,主要是基于代码的可读性和可维护性考虑,
当然是在其性能开销可以接受的情况下。

\useurl[msmVSsc][http://stackoverflow.com/questions/4275602/boost-statechart-vs-meta-state-machine]
\useurl[sc_perf][http://www.boost.org/doc/libs/1_53_0/libs/statechart/doc/performance.html]
\startnotepar
\noindent MSM 和 statechart 的比较:

\from[msmVSsc]

\noindent statechart 的性能:

\from[sc_perf]
\stopnotepar

\section{业务模型的物理视图}
每一项业务的内部结构如\reffig{fs_svc_phy_view}所示:



\startreusableMPgraphic{fsSvcArch}

vardef itemSvc(expr lbl) =
image(
	draw txtFrame(12u,12v, btex \mplabel{} etex);
	label.lrt(lbl, (-6u,6v));
	draw txtFrame(6u, 1.5v, btex \mplabel{Server} etex) shifted (0, 6v);
	draw txtFrame(10u, 1.5v, btex \mplabel{控制器} etex) shifted (0, 3v);
	draw txtFrame(4u, 1.5v, btex \mplabel{状态机} etex);
	draw txtFrameDashed(6u, 1.5v, btex \mplabel{上下文} etex) shifted (0, -2v);
	draw txtFrame(4u, 1.5v, btex \mplabel{设备...} etex) shifted (-3u, -6v);
	draw txtFrame(4u, 1.5v, btex \mplabel{配置数据} etex) shifted (3u, -5v);

	draw itemArrow((-4u,-5.5v), (-4u,2.5v));	% device -> controller
	draw itemArrow((4u,2.5v), (4u,-4.5v));		% controller -> cfg
	draw itemArrow((-2u,3.5v), (-2u,5.5v));		% controller -> server
	draw itemArrow((2u,5.5v), (2u,3.5v));		% server -> controller
	draw itemArrow((-u,.5v), (-u,2.5v));		% sm -> controller
	draw itemArrow((u,2.5v), (u,.5v));		% controller -> sm

	draw itemArrow((-u,-0.5v), (-2u,-5.5v));	% sm -> controller
	draw itemArrow((u,-0.5v), (2u,-4.5v));		% controller -> cfg
)
enddef;


draw itemSvc(btex \mplabel{业\\务\\N} etex);

% legend
vardef itemLengend =
image(
	label.rt(btex \mplabel{图例:} etex, (0,0)) shifted(-13u,0);

	draw itemArrowRO((-10u, 0), (-9u,0));
	label.rt(btex \mplabel{只读} etex, (0,0)) shifted (-9u, 0);

	draw itemArrowRW((-6u, 0), (-5u,0));
	label.rt(btex \mplabel{读写} etex, (0,0)) shifted (-5u, 0);

	draw itemArrow((-2u, 0), (-1u,0));
	label.rt(btex \mplabel{控制} etex, (0,0)) shifted (-1u, 0);
)
enddef;

%draw itemLengend shifted (0, -7v);

\stopreusableMPgraphic


\placefigure[here][fig:fs_svc_phy_view]
{业务模型的物理视图}
{\reuseMPgraphic{fsSvcArch}}

\subsection{控制器}
「控制器」是整个业务的核心,本质是一个 poller (参见\refsec{evt_drv}),
主要功能就是获取外部消息,
如 socket 消息(来自「Server」)、设备中断、定时器超时等等。
「控制器」收到消息后,如果发现是要通知「状态机」的,
则将消息转换成「状态机」所能识别的事件,并发送给「状态机」;否则就地处理。
另外「控制器」也会往外发送消息(通过「Server」),比如报告状态。
\subsection{Server}
「Server」是整个业务与其他模块交流的主要通道,一个业务本质是要提供一个 Service,
这里将「Server」实现为一个 TCP Server,其他模块可以给其发送消息,
比如状态查询、参数设置、业务启停等。所有消息都由「控制器」接收处理。
而「控制器」也可以通过「Server」给其他模块发送消息,比如报告状态,返回查询结果等。
\subsection{上下文}
「上下文」主要存放的是业务相关设备文件和「配置数据」,但都是引用,
这部分内容主要是给「状态机」用。
「上下文」的存在主要是为了限制状态机对这些数据的访问,
比如「配置数据」,不允许「状态机」对其更改,但其原始数据可以由「控制器」(处理配置消息时)更改,
其改动在「状态机」访问「上下文」的时候也可以体现。
\subsection{状态机}
「状态机」用于维护业务流的状态,接收来自「控制器」的事件,
并根据自身的状态来操控「上下文」中的设备,比如启动电机、启动高压板等;
然后完成自身状态的转换。

\startnotepar
「设备...」中除了「设备」还有「状态机」会用到的其他 pollee,如定时器。
\stopnotepar

下面介绍熔接机中一些主要业务的实现,每一种业务均从三方面进行介绍:
\startigBase
\item 消息,与其他模块交互用。
\item 流程,业务的主要流程。
\item 类图,业务的具体实现。
\stopigBase

%\color[red]{\bfa\textbackslash\textbackslash TODO: 请添加相关内容。}
\stopcomponent
