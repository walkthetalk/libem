\startcomponent cpn-svcFS
\product fusion-splicer-SD

\chapter{业务:熔接}

熔接是熔接机的核心业务,也是熔接机所有业务中涉及器件最多、最复杂的,本节主要介绍此业务的实现。

\section{消息}
熔接业务主要与 UI 部分进行交互,所使用的消息包括:
\startigBase
\item fs_start:来自 UI,用于启动熔接;
\item fs_continue:来自 UI (源自按键),继续熔接;
\item fs_stop:来自 UI,用于停止熔接;
\item fs_state:发往 UI,用于报告状态。
\stopigBase
目前实现中业务所需的配置数据,如熔接参数、操作选项等,均来由 fs_start 消息,
将来可以添加一个不带配置数据的 start 消息,这样只有第一次启动熔接的时候需要发送 fs_start,
后面如果配置数据没有变化,就只需发送不带配置数据的 start 消息即可。

\section{流程}

\reffig{fs_sm}是熔接业务的主要流程图。

\placefigure[here][fig:fs_sm]
{熔接业务流程图}
{\externalfigure[mp/umls-2.pdf]}

为了便于理解,\reffig{fs_sm}中只画出了主要流程。
由\reffig{fs_sm}可以看出,熔接业务主要有两大状态,一为「idle」,即空闲态;
一为「running」,即运行态。

「idle」是熔接业务的初始状态。目前此状态下主要处理的事件是 evStart,
由此业务的控制器根据消息 fs_start 转换而来。
当收到此事件后,转变为「running」状态。

「running」状态中的所有子状态都可以看作是熔接的一个步骤,这些步骤在正常情况下是顺序跳转的。
对于任何一个步骤而言,
如果收到了 evStop 事件(由控制器根据消息 fs_stop 转换而来),
或者出现了错误则跳转到「idle」态,
如果是出现了错误,还要通过控制器发送消息 fs_state 上报错误讯息。
熔接结束也会跳转到「idle」态,并上报状态。
此状态下所处理的事件列表:
\startigBase
\item evStop:停止熔接,收到此事件后会转入「idle」态,并执行一些初始化操作,比如电机复位。
\item evContinue:继续熔接,由消息 fs_continue 转换而来,
当子状态为「pause1」和「pause2」时接收到此事件会跳转到下一个状态。
\item evEntryAct:此事件为状态机内部事件,由状态机进入某一个状态时自发自收,
比如,进入「push1」状态后就会发送此事件,接收到此事件后会检查光纤间距是否达到了预设值,
如果是,则跳转到下一状态,否则操控电机。
\item evMotorStop:电机停止事件,此事件由电机中断出发,代表电机运行完成,可以继续操控电机了。
\item evHvbTimeout:高压板放电时间已经达到了预设值,收到此事件后关闭高压板,不再放电。
\stopigBase

\startnotepar
控制器在接收到配置数据后需要判断是否位于「running」态,如果是,则丢弃配置数据并回告错误讯息。
即熔接业务运行时不允许修改配置数据。

另外,「running」中的子状态发生变化时(正常流程)是否需要上报?
如果 UI 上除图像外还要以文字的形式显示当前状态,则需要上报,暂时不上报,如有需要再添加。
\stopnotepar

\section{类图}

\reffig{fs_class_diag} 是熔接业务的类图(仅包括主要类)。
\placefigure[here][fig:fs_class_diag]
{熔接业务类图}
{\externalfigure[mp/fs_class_diag-1.pdf]}
由图中可以看出,熔接业务主要由三个类构成:
\startigBase
\item svcFusionSplicing:是此业务的核心类,继承自执行模型中的「poller」,
相当于业务的「控制器」;
\item fs_sm:是此业务的「状态机」;
\item fs_ctx:是此业务的「上下文」。
\stopigBase
下面分别介绍这三个类(只由代码主干,其他以 \ccmm{...} 取代)。

\subsection{类: svcFusionSplicing}

\startCPP
class svcFusionSplicing : public exemodel::poller {
	...
private:
	/// 处理 poller 由 m_svr 收到的消息
	void __svr_cb(exemodel::serveree::args_t & args);
	/// 分别处理四个电机的中断
	void __motorLX_cb(motor::args_t & args);
	void __motorLY_cb(motor::args_t & args);
	void __motorRX_cb(motor::args_t & args);
	void __motorRY_cb(motor::args_t & args);
private:
	/// 状态机用此函数上报状态
	void __report(const std::string & info);
private:
	exemodel::serveree m_svr;	/// server,用于与其他模块通信
	msg::recver m_recver;		/// 消息接收器,与 m_svr 配合使用
	msg::sender m_sender;		/// 消息发送器,与 m_svr 配合使用

	camera	m_camera;	/// 摄像头
	motor	m_motorLX;	/// 左侧 X 轴电机
	motor	m_motorLY;	/// 左侧 Y 轴电机
	motor	m_motorRX;	/// 右侧 X 轴电机
	motor	m_motorRY;	/// 右侧 Y 轴电机
	hvb	m_hvb;		/// 高压板

	cfg_data_t m_cfg;	/// 配置数据

	svcFS::fs_ctx m_ctx;	/// 上下文
	svcFS::fs_sm m_sm;	/// 状态机
};
\stopCPP

此类在构造函数中会将摄像头、电机、高压板、配置数据等的引用添加到 \cvar{m_ctx} 中,
将 \cvar{__report} 和 \cvar{m_ctx} 都传递给 \cvar{m_sm}。

\capi{__svr_cb} 中会处理收到的消息,并将其转换成事件发送给 \cvar{m_sm}。

\subsection{类: fs_ctx}

\startCPP
class fs_ctx {
	...
public:
	camera	& m_camera;		/// 摄像头
	motor	& m_motorLX;		/// 左侧 X 轴电机
	motor	& m_motorLY;		/// 左侧 Y 轴电机
	motor	& m_motorRX;		/// 右侧 X 轴电机
	motor	& m_motorRY;		/// 右侧 Y 轴电机
	hvb	& m_hvb;		/// 高压板
	const cfg_data_t & m_cfg;	/// 配置数据
};
\stopCPP

此类中主要存储的是相关设备以及配置数据的引用,注意配置数据时 \cqlf{const} 引用,
即状态机在使用此类时不能修改配置数据。为了方便状态机的使用,本来可以为此类添加一些接口,
但考虑到控制的复杂性,可能要添加的接口很多,因此所有引用都做成 \ckey{public} 的,
由状态机直接访问。

\subsection{类: fs_sm}

\startCPP
struct stIdle;	/// 初始状态
struct fs_sm : boost::statechart::state_machine< fs_sm, stIdle > {
	...
public:
	fs_ctx & m_ctx;		/// 上下文的引用
	std::function<void(const std::string & )> m_reporter;	/// 用于上报状态的回调
};
\stopCPP

状态机里面主要由两项内容:一是上下文的引用,一是用于上报状态的回调,均在构造时确定。

\reftab{fs_evt_list}中列出了熔接业务状态机所处理的所有事件及相应的类名。
\reftab{fs_state_list}中列出了熔接业务状态机中的所有状态及相应的类名。

\placetable[here][tab:fs_state_list]{熔接业务状态列表}{

\bTABLE
\setupTABLE[c][1][align={flushright,lohi}]
\setupTABLE[c][2][align={flushleft,lohi}]

\bTABLEhead
\bTR
\bTD 类名 \eTD\bTD 简介 \eTD
\eTR
\eTABLEhead

\bTABLEbody

\bTR
\bTD \ctype{stIdle} \eTD\bTD 空闲态 \eTD
\eTR
\bTR
\bTD \ctype{stRunning} \eTD\bTD 运行态 \eTD
\eTR
\bTR
\bTD \ctype{stEntering} \eTD\bTD 光纤还未进入视野 \eTD
\eTR
\bTR
\bTD \ctype{stPush1} \eTD\bTD 推进光纤达到预设间隙 1 \eTD
\eTR
\bTR
\bTD \ctype{stClearing} \eTD\bTD 清洁放电 \eTD
\eTR
\bTR
\bTD \ctype{stDefectDetecting} \eTD\bTD 缺陷检测 \eTD
\eTR
\bTR
\bTD \ctype{stPush2} \eTD\bTD 推进光纤达到预设间隙 2 \eTD
\eTR
\bTR
\bTD \ctype{stPause1} \eTD\bTD 暂停 1 \eTD
\eTR
\bTR
\bTD \ctype{stCalibrating} \eTD\bTD 调芯对准 \eTD
\eTR
\bTR
\bTD \ctype{stPause2} \eTD\bTD 暂停 2 \eTD
\eTR
\bTR
\bTD \ctype{stPreSplicing} \eTD\bTD 预熔 \eTD
\eTR
\bTR
\bTD \ctype{stSplicing} \eTD\bTD 熔接 \eTD
\eTR
\bTR
\bTD \ctype{stLossEstimating} \eTD\bTD 损耗估算 \eTD
\eTR
\bTR
\bTD \ctype{stStoring} \eTD\bTD 存储 \eTD
\eTR

\eTABLEbody

\eTABLE

}
\placetable[here][tab:fs_evt_list]{熔接业务事件列表}{

\bTABLE
\setupTABLE[c][1][align={flushright,lohi}]
\setupTABLE[c][2][align={flushleft,lohi}]

\bTABLEhead
\bTR
\bTD 类名 \eTD\bTD 简介 \eTD
\eTR
\eTABLEhead

\bTABLEbody

\bTR
\bTD \ctype{evStart} \eTD\bTD 启动熔接 \eTD
\eTR
\bTR
\bTD \ctype{evStop} \eTD\bTD 停止熔接 \eTD
\eTR
\bTR
\bTD \ctype{evContinue} \eTD\bTD 继续熔接 \eTD
\eTR
\bTR
\bTD \ctype{evEntryAct} \eTD\bTD 状态入口事件 \eTD
\eTR
\bTR
\bTD \ctype{evMotorStop} \eTD\bTD 电机停止 \eTD
\eTR
\bTR
\bTD \ctype{evHvbTimeout} \eTD\bTD 高压板放电完毕 \eTD
\eTR
\eTR

\eTABLEbody

\eTABLE

}

下面以 \ctype{stPush1} 的代码为例介绍状态机的执行方式:
\subsubsection{状态: push1}
\startCPP
struct stPush1 : sc::state< stPush1, stRunning > {
	/// 此状态下可以处理下列三种事件:
	typedef boost::mpl::list<
		sc::custom_reaction< evEntryAct >,
		sc::custom_reaction< evMotorStop<motorId_t::LX> >,
		sc::custom_reaction< evMotorStop<motorId_t::RX> >
	> reactions;
public:
	stPush2(my_context ctx)
	: my_base(ctx)
	{
		/// 进入此状态后先自发自收事件 evEntryAct
		post_event(evEntryAct());
	}

	sc::result react(const evEntryAct &)
	{
		if (两根光纤均已到位) {
			/// 跳转到清洁放电状态
			return transit<stClearing>();
		}

		if (左侧光纤没有到位) {
			/// \todo 操控左侧 X 轴电机
		}

		if (右侧光纤没有到位) {
			/// \todo 操控右侧 X 轴电机
		}

		/// 维持此状态不变
		return discard_event();
	}

	sc::result react(const evMotorStop<motorId_t::LX> &)
	{
		if (左侧光纤没有到位) {
			/// \todo 操控左侧 X 轴电机
		}

		if (右侧光纤已经到位) {
			/// 此时两根光纤均已到位
			/// 跳转到清洁放电状态
			return transit<stClearing>();
		}

		/// 维持此状态不变
		return discard_event();
	}

	sc::result react(const evMotorStop<motorId_t::RX> &)
	{
		if (右侧光纤没有到位) {
			/// \todo 操控右侧 X 轴电机
		}

		if (左侧光纤已经到位) {
			/// 此时两根光纤均已到位
			/// 跳转到清洁放电状态
			return transit<stClearing>();
		}

		/// 维持此状态不变
		return discard_event();
	}
};
\stopCPP

\startnotepar
可能需要添加定时器来检查电机是否已经停止运行,以避免丢中断所带来的问题。
\stopnotepar

\stopcomponent
