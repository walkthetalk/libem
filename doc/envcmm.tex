%
% author:	Ni Qingliang
% date:		2011-02-11
%
\startenvironment envcmm

% colors
%\setupcolors[state=start]
\setupcolor[dem]
%%%%%%%%%%%%%%%%%%% simplefonts %%%%%%%%%%%%%%%%%%%%%%%%%%%%%%%%%%%%%%%%%%%%%%%%
%% use `mtxrun --script font --list --all` to list all fonts
% only MKII need specify encoding,
% it is UTF-8 defaultly in MKIV
% \enableregime[utf]
% \usemodule[chi-00]
%\mainlanguage[cn]

%%%%%%%%%%%%%%%%%%% zhfonts     %%%%%%%%%%%%%%%%%%%%%%%%%%%%%%%%%%%%%%%%%%%%%%%%
% 本來應該是 9pt,但是由於目前context有bug,math模式內的text的字體設定有問題,此處暫時用 9.1pt
\usemodule[zhfonts][style=rm, size=9.1pt]
%\setupzhfonts[feature][onum=yes, pnum=yes]

%% use `mtxrun --script font --list --all` to list all fonts
% set base fonts
\setupzhfonts
[serif]
[regular=adobesongstd,
bold=adobeheitistd,
italic=adobekaitistd,
bolditalic=adobeheitistd]

\setupzhfonts
[sans]
[regular=adobefangsongstd,
bold=adobeheitistd,
italic=adobefangsongstd,
bolditalic=adobeheitistd]

\setupzhfonts
[mono]
[regular=adobekaitistd,
bold=adobeheitistd,
italic=adobekaitistd,
bolditalic=adobeheitistd]

% set latin fonts
\setupzhfonts
[latin, serif]
[regular=texgyrepagellaregular,
bold=texgyrepagellabold,
italic=texgyrepagellaitalic,
bolditalic=texgyrepagellabolditalic]

\setupzhfonts
[latin, mono]
[regular=dejavusansmono,
bold=dejavusansmonobold,
italic=dejavusansmonoboldoblique,
bolditalic=dejavusansmonooblique]

\setupzhfonts
[latin, sans]
[regular=texgyreherosregular,
bold=texgyreherosbold,
italic=texgyreherositalic,
bolditalic=texgyreherosbolditalic]

%\setupzhfonts[math][roman=]

% 启用中文断行
\setscript[hanzi]
\mainlanguage[cn]

%%%%%%%%%%%%%%%%%%%%%%%%%%%%%%%%%%% interaction %%%%%%%%%%%%%%%%%%%%%%%%%%%%%%%%
%\startallmodes[screen,solution]
\setupinteraction[
  state=start,	%start stop
  focus=standard,
  %menu	on off
  %page	yes no
  %click	yes no
  %split	yes no
  %display	new
  %openaction	reference
  %closeaction	reference
  %openpageaction	reference
  %closepageaction	reference
  %calculate	name
  %strut	yes no
  %width	dimension
  %height	dimension
  %depth	dimension
  style=\ftRef,	%normal bold slanted boldslanted type cap small...command
  color=black,	%name	link to other page
  contrastcolor=black,	%name	link to the same page
  %symbolset	name
  %title	text
  %subtitle	text
  %author	text
  %date	text
  %keyword	text
  %fieldlayer	auto name
]
%\stopallmodes

\setuptolerance[horizontal,
  verystrict, %stretch space verystrict strict tolerant verytolerant
]
\setuptolerance[vertical,
  strict, %stretch space verystrict strict tolerant verytolerant
]
%%%%%%%%%%%%%%%%%%%%%%%%%%%%font%%%%%%%%%%%%%%%%%%%%%%%%%%%%%%%%%%%%%%%%%%%%%%%%
\define\ftEmp{\bf} % emphasize
\define\ftRef{\it} % reference
\definetype[capi][ % api
  %space	on off
  %option	slanted normal none
  style=\tt\bf,	%normal bold slanted boldslanted type cap small ... command
  %color	name
]
\define[1]\mapi{%
\text{\capi{#1}}%
}

\definetype[capiemp][
  style=\tt\bf,
]
\define[1]\mapiemp{%
\text{\capiemp{#1}}%
}

\definetype[carg][ % argument
  style=\tt\it,
]
\define[1]\marg{%
\text{\carg{#1}}%
}

\definetype[ctype][ % type
  style=\tt\tf,
]
\define[1]\mtype{%
\text{\ctype{#1}}%
}

\definetype[ctypeemp][ % type
  style=\tt\bf,
]

\definetype[cenum][ % enumeration
  style=\tt\tf,
]
\define[1]\menum{%
\text{\cenum{#1}}%
}

\definetype[cenumemp][ % type
  style=\tt\bf,
]

\definetype[cqlf][ % qualifier
  style=\tt\it,
]
\definetype[cqlfemp][
  style=\tt\bf,
]
\definetype[cdrt][ % directive
  style=\tt\tf,
]
\definetype[cdrtemp][
  style=\tt\bf,
]

\definetype[cmacro][
  style=\tt\tf,
]
\define[1]\mmacro{%
\text{\cmacro{#1}}%
}

\definetype[cmacroemp][
  style=\tt\bf,
]
\definetype[cpragmaemp][
  style=\tt\bf,
]

\definetype[cvar][ % variable
  style=\tt\tf,
]
\define[1]\mvar{%
\text{\cvar{#1}}%
}

\definetype[ckey][
  style=\tt\tf,
]

\definetype[ccmm][ % common
  style=\tt\tf,
]
\define[1]\mcmm{%
\text{\ccmm{#1}}%
}

\definetype[ccmmsuffix][
  style=\tt\it,
]
\definetype[cemp][
  style=\tt\bf,
]
\definetype[cempsuffix][
  style=\tt\bi,
]
\definetype[cref][
  style=\tt\it,
]
\definetype[cfmt][% format
  style=\tt\it,
]
\definetype[clext][ % opencl extension
  style=\rm\bf,
]

%%%%%%%%%%%%%%%%%%%%%%%%%%%%%%%%%%layout%%%%%%%%%%%%%%%%%%%%%%%%%%%%%%%%%%%%%%%%

\definepapersize[CD][width=20cm,height=20cm]
\setuppapersize[A4][A4]

\setuppagenumbering[
  state=start,
  way=bytext,
  alternative=doublesided,
  location=,%header,
  %left=the,
  %right=ok,
]
% don't use TAB, and no space between *reg=val*, and no null line,
% or it will be ommitted
\setuplayout[
  % vertical
  top=20mm,
    topdistance=2mm,
      header=\bodyfontsize,
        headerdistance=2mm,
        footerdistance=2mm,
      footer=\bodyfontsize,
    bottomdistance=2mm,
  bottom=20mm,
  topspace=\dimexpr(\topheight + \topdistance),
  bottomspace=\dimexpr(\bottomheight + \bottomdistance),
  height=fit,
  % horizontal
  leftedge=10mm,
    leftedgedistance=2mm,
      leftmargin=20mm,
        leftmargindistance=2mm,
        rightmargindistance=2mm,
      rightmargin=20mm,
    rightedgedistance=2mm,
  rightedge=10mm,
  backspace=\dimexpr(\leftedgewidth + \leftedgedistance + \leftmarginwidth + \leftmargindistance + 5mm),
  cutspace=\dimexpr(\rightedgewidth + \rightedgedistance + \rightmarginwidth + \rightmargindistance),
  width=fit,
  % misc
  location=middle,
  marking=on
]

% set the background clors
% NOTE: it is confilict with \show frame
\if 0
\setupbackgrounds[page][background=color,backgroundcolor=gray:1]
\setupbackgrounds[top, bottom][background=color,backgroundcolor=gray:1]
\setupbackgrounds[header][background=color,backgroundcolor=darkyellow]
\setupbackgrounds[footer][background=color,backgroundcolor=darkyellow]
\setupbackgrounds[text][leftedge,leftmargin,text,rightmargin,rightedge][background=color,backgroundcolor=white]
\fi

%%%%%%%%%%%%%%%%%%%%%%%%%%%%%setup list%%%%%%%%%%%%%%%%%%%%%%%%%%%%%%%%%%%%%%%%%
\definesorting[sClFunc][sClFuncs]

\define[1]\topclfunc{%
\sClFunc{#1}%
\reference[sClFunc:#1]{#1}%
\hskip-2em\framed[%
  width=\overlaywidth,
  frame=off,
  bottomframe=on,
  align=flushleft,
  before={\blank[0.5ex]},
  after={},
  framecolor=darkgray,
  rulethickness=2pt,
]{#1}\par%
}

% number~name~name
\define[3]\listtopclfunc{%
\clapi{#2} \at[sClFunc:#3]\par%
}

\setupsorting[sClFunc][%
  criterium=all,	% all used
  %before=,	%COMMAND
  %after=,	%COMMAND
  command=\listtopclfunc,	%\...#1#2#3
  %state=,	%start stopallmodes
  %style=,	%normal bold slanted boldslanted type cap small... COMMAND
  %expansion=,	%yes no command
]

%%%%%%%%%%%%%%%%%%%%%%%%%%%%footnote%%%%%%%%%%%%%%%%%%%%%%%%%%%%%%%%%%%%%%%%%%%%
\setupfootnotes[
  %conversion=set 2,	%numbers characters Characters romannumerals Romannumerals
  %way=bytext,	%bytext bysection
  %location=high,	%page text columns firstcolumn lastcolumn high none
  %background=color,
  %backgroundcolor=red,
  %rule	on off
  %before=,	%command
  %after	command
  %width	dimension
  %height	dimension
  %bodyfont	5pt ... 12pt small big
  %style=bold,	%normal bold slanted boldslanted type cap small... command
  distance=2ex,	%dimension
  %columndistance	dimension
  %margindistance	dimension
  %n=3,		%number
  %numbercommand=\high,	%oneargument 脚注编号的位置
  %textcommand=,	%oneargument
  %split	tolerant strict verystrict number
  %textstyle	normal bold slanted boldslanted type cap small... command
  %textcolor=red,	%name
  %interaction=yes,	%yes no
  %factor	number
]

%%%%%%%%%%%%%%%%%%%%%%%%%%%% inmargin %%%%%%%%%%%%%%%%%%%%%%%%%%%%%%%%%%%%%%%%%%
\setupinmargin[%
  %location %flushleft flushright both
  style={\rm\itx},	%normal bold slanted boldslanted type cap small... command
  %before	%command
  %after	%command
  align=left,	%inner outer left right middle normal no yes
  %line	number
  %distance	dimension
  %separator	text
  %width	dimension
  %distance	dimension
  %stack	yes no
  %see \setupframed
]
%%%%%%%%%%%%%%%%%%%%%%%%%%%% paragraph %%%%%%%%%%%%%%%%%%%%%%%%%%%%%%%%%%%%%%%%%

% 首行缩进
\setupindenting[first,always,2em]
% 段落间距
\setupwhitespace[small] %none small medium big line fixed fix dimension
% 行间距
\setupinterlinespace[medium]
%\setupinterlinespace[big, on][line=3ex]
%\setuprelativeinterlinespace[1.2]
% setupalign 必須位於setupbodyfont之前,因此修改了zhfonts模塊
%\setupalign[flushleft,nothyphenated,broad]
%width left right middle inner outer wide broad height bottom line reset hanging
%nothanging hyphenated nothyphenated lesshyphenation morehyphenation new old normal yes
%no flushleft flushright flushouter flushinner center hz nohz spacing nospacing tolerant
%verytolerant stretch

% code
%\installprettytype[C][C]
\setuplinenumbering[
% conversion	numbers characters Characters romannumerals Romannumerals text
% start	number
% step	number
%  width=1em,	%dimension
%  location=text,	%intext inmargin
% style	normal bold slanted boldslanted type cap small... command
% prefix	text
% referencing	on off
]
\definetextbackground[verb][
  frame=off,
  location=paragraph,
  leftoffset=0.5em,
  rightoffset=0.5em,
  topoffset=0.25em,bottomoffset=0.25em,
  rulethickness=0.75pt,
  strut=no,
  backgroundcolor={gray:1},
]

\setuptyping[
  escape=yes,	%character %{[[,]]}
  %space	on off
  tab=8,	%number yes no
]

\definetyping[CPP][%option=c,
  bodyfont=7pt,
  %space	on off
  %page	yes no
  %option	slanted normal commands color none
  %text	yes no
  %icommand	command
  %vcommand	command
  %ccommand	command
  before={\vskip.5ex\starttextbackground[verb]},	%command
  after={\stoptextbackground\vskip.5ex},	%command
  margin=no,	%dimension standard yes no
  %evenmargin	dimension
  %oddmargin	dimension
  %blank	dimension small medium big standard halfline line
  escape=yes,	%character %{[[,]]}
  %space	on off
  tab=8,	%number yes no
  %page	yes no
  %indentnext	yes no
  style=\tt\tf,		%normal bold slanted boldslanted type cap small... command
  %color	name
  %palet	name
  %lines=yes,	%yes no hyphenated
  %empty	yes all no
  numbering=line,	%line file no
]

% c in table
\definetyping[cintbl][
  option=none,
  before={\blank},
  after={\blank},
  tab=8,
]

% note
\defineframedtext[notepar][
  bodyfont=small,	%5pt ... 12pt small big
  style=\ftRef, %normal bold slanted boldslanted type cap small... COMMAND
  %left=,	%COMMAND
  %right,	%COMMAND
  %before=,	%COMMAND
  %after,	%COMMAND
  %inner,	%COMMAND
  %linecorrection=,	%on off
  %depthcorrection=,	%on off
  %margin=,	%standard yes no
  %location=,	%left right middle none
  indenting={first,always,2em},	%never none not no yes always first next small medium big normal odd even DIMENSION
  %inherits from \setupframed
  background=screen,
  frame=off,
  rightframe=on,
  leftframe=on,
  framecolor=darkgreen,
  rulethickness=3pt,
  width=local,
]

% note
\defineframedtext[replacepar][
  bodyfont=small,	%5pt ... 12pt small big
%  style=\ftRef, %normal bold slanted boldslanted type cap small... COMMAND
  %left=,	%COMMAND
  %right,	%COMMAND
  %before=,	%COMMAND
  %after,	%COMMAND
  %inner,	%COMMAND
  %linecorrection=,	%on off
  %depthcorrection=,	%on off
  %margin=,	%standard yes no
  %location=,	%left right middle none
  indenting={first,always,2em},	%never none not no yes always first next small medium big normal odd even DIMENSION
  %inherits from \setupframed
  background=screen,
  frame=off,
  rightframe=on,
  leftframe=on,
  framecolor=darkgreen,
  rulethickness=3pt,
  width=local,
]

%%%%%%%%%%%%%%%%%%%%%%%%%%%%enum%%%%%%%%%%%%%%%%%%%%%%%%%%%%%%%%%%%%%%%%%%%%%%%%

\defineenumeration[example][
  text={例},
  headstyle=\rm\bf,	%normal bold slanted boldslanted type cap small... command
%  headcolor=blue:7,
%style	normal bold slanted boldslanted type cap small... command
%color	name
%width	fit broad dimension
%distance	dimension
%sample	text
%text	text
%align	flushleft middle flushright
%margin	standard yes no dimension
  alternative=top,	%left right top serried inmargin inleft inright hanging
%headcommand	command
%hang	fit broad number
%before	command
%inbetween	command
%after	command
%indentnext	yes no
%indenting	never not no yes always first next
  prefix=yes,
  prefixsegments=chapter, %chapter:section
]

\defineenumeration[QUESTION][
  text={問題},
  right={:},
  headstyle=\rm\bf,	%normal bold slanted boldslanted type cap small... command
  headcolor=red,
  %style=\rm\bf,		%normal bold slanted boldslanted type cap small... command
  color=red,	%name
  width=fit,	%fit broad dimension
  distance=0em,		%dimension
%sample	text
%text	text
  %align=flushleft,	%flushleft middle flushright
  %margin=no,	%standard yes no dimension
  alternative=serried,	%left right top serried inmargin inleft inright hanging
  %hang=2,		%fit broad number
%headcommand	command
%before	command
%inbetween	command
%after={},	%command
%indentnext	yes no
%indenting	never not no yes always first next
  %prefix=no,
  %prefixsegments=chapter, %chapter:section
]

\definestartstop[ANSWER]

% no number
\defineitemgroup[igBase][levels=2]
\setupitemgroup[igBase]
[1]
[packed,joinedup,]
%standard broad serried packed unpacked stopper joinedup atmargin inmargin autointro loose repeat section paragraph intext random columns
%standard: default setup
%n*broad:  extra horizontal white space after symbol
%n*serried:little horizontal white space after symbol
%packed:   no whitespace between items
%stopper:  punctuation after item separator
%joinedup: no white space before and after itemization
%atmargin: item separator at the margin
%inmargin: item separator in margin
[
%margin	no standard dimension
  leftmargin=2em,	%no standard dimension
%rightmargin	no standard dimension
%width	dimension
%distance	dimension
%factor	number
%items	number
%start	number
  before=,	%command
  inbetween=,	%command
  after={\blank[.5ex]},	%command
%left	text
%right	text
%beforehead	command
%afterhead	command
%headstyle=boldslanted, %normal bold slanted boldslanted type cap small... command
%marstyle=boldslanted, %normal bold slanted boldslanted type cap small... command
%symstyle=boldslanted, %normal bold slanted boldslanted type cap small... command
%stopper	text
%n	number
%symbol	number
%align	left right normal
%indentnext	yes no
]
\setupitemgroup[igBase][2]
[packed,joinedup,][
  leftmargin=4em,	%no standard dimension
  before=,	%command
  inbetween=,	%command
  after=,	%command
]
\setupitemgroup[igBase][1][1] % 枚举的标识必须单独定义,否则无效,可选的
\setupitemgroup[igBase][2][2]
%m	A numbered list, with lowercase (“medieval”, aka “oldstyle”) numbers.
%1 … 8	Different kinds of bullets. All items get the same symbol.
%a	Items are numbered a., b., c., …
%A	Items are numbered A., B., C., …
%AK	Items are numbered A., B., C., …, in small caps.
%r	Items are numbered in lowercase Roman numerals.
%R	Items are numbered in uppercase Roman numerals.
%KR	Items are numbered in uppercase Roman numerals, small caps style.


% with number
\defineitemgroup[igNum][levels=1]
\setupitemgroup[igNum]
[each]
[packed,joinedup,]
[
  leftmargin=2em,	%no standard dimension
  before=,	%command
  inbetween=,	%command
  after={\blank[.5ex]},	%command
]
\setupitemgroup[igNum][1][n]

% big item
\defineitemgroup[igBig][levels=2]
\setupitemgroup[igBig][1]
[packed,joinedup,]
[
  leftmargin=0em,
  before=,
  inbetween=,
  after={\blank[.5ex]},
  width=2em,
]
\setupitemgroup[igBig][1][a]
\setupitemgroup[igBig][2]
[packed,joinedup,]
[
  leftmargin=2em,
  before=,
  inbetween=,
  after=,
  width=2em,
]
\setupitemgroup[igBig][2][1]

%%%%%%%%%%%%%%%%%%%%%%%%%%table%%%%%%%%%%%%%%%%%%%%%%%%%%%%%%%%%%%%%%%%%%%%%%%%%
\setupTABLE[%
    %frameoffset=.5\linewidth,
    %backgroundoffset=\v!frame,
    %framecolor=\s!black,
    %width=\v!fit,
    %height=\v!fit,
    %autowidth=\v!yes,
    %rulethickness=\linewidth,
    %strut=\v!yes,
    %autostrut=\v!no,
    %
    %color=,
    style={\rmx},
    headstyle={\rmx\bf},
    %headcolor=,
    %aligncharacter=\v!no,
    %alignmentcharacter={,},
    %option=, % \v!stretch
    %header=,
    %spaceinbetween=,
    %maxwidth=8em,
    %textwidth=\hsize,
    split=yes,	% repeat
    %splitoffset=0pt,
    %distance=\zeropoint,           % individual column
    %columndistance=\zeropoint,     % each column (whole table)
    %leftmargindistance=\zeropoint, % whole table
    %rightmargindistance=\zeropoint,% whole table
    %left=,
    %right=,
    %setups=,
    splitmethod=b%
]
\setupTABLE[header][
    style={\rmx\bf},
    background=color,
    backgroundcolor=gray:1,
]

%%%%%%%%%%%%%%%%%%%%%%%%%%shotcut%%%%%%%%%%%%%%%%%%%%%%%%%%%%%%%%%%%%%%%%%%%%%%%
%% TODO
%%%%%%%%%%%%%%%%%%%%%%%%%% description %%%%%%%%%%%%%%%%%%%%%%%%%%%%%%%%%%%%%%%%%
%% clOption
\definedescription[clOption][
  %headstyle	normal bold slanted boldslanted type cap small... command
  %style	normal bold slanted boldslanted type cap small... command
  %color	name
  %width	fit broad dimension
  %distance	dimension
  %sample	text
  %text	text
  %align	flushleft middle flushright
  margin=2em,	%standard yes no dimension
  alternative=top,	%left right top serried inmargin inleft inright hanging
  headcommand=\hskip-2em,	%command
  %command=\hskip-2em,	%command
  %hang=,	%fit broad number
  before=\vskip.5em,		%command
  inbetween=\vskip.0em,		%command
  after=\vskip.5em,		%command
  %indentnext	yes no
  indenting=yes,	%never not no yes always first next
]
%% clSpecifier
\definedescription[clSpecifier][
  width=3em,
  distance=1em,
  margin=2em,	%standard yes no dimension
  alternative=hanging,	%left right top serried inmargin inleft inright hanging
  before=\vskip.5em,		%command
  after=\vskip.5em,		%command
  indenting=yes,
]
%% clCmmDesc
\definedescription[clCmmDesc][
  headstyle=\tt\tf,
  style=\tt\tf,
  margin=2em,	%standard yes no dimension
  alternative=top,	%left right top serried inmargin inleft inright hanging
  headcommand=\hskip-2em,	%command
  before=\vskip.25em,		%command
  inbetween=\vskip.0em,		%command
  after=\vskip.25em,		%command
  indenting=yes,	%never not no yes always first next
]
%%%%%%%%%%%%%%%%%%%%%%%%%%%%%%%%%%%%%%%%%%%%%%%%%%%%%%%%%%%%%%%%%%%%%%% glossary
\definesynonyms[sClGlo][sClGlos][\englo][\cnglo]

\definedescription[clglodsc][
  headstyle={\rm\bfa},	%normal bold slanted boldslanted type cap small... command
  %style=normal,	%normal bold slanted boldslanted type cap small... command
  %color=,		%name
  %width=fit,		%fit broad dimension
  %distance=,		%dimension
  %sample=,		%text
  %text=,		%text
  %align=,		%flushleft middle flushright
  %margin=2em,		%standard yes no dimension
  alternative=top,		%left right top serried inmargin inleft inright hanging
  %headcommand=\vskip.0em,		%command % ERROR: with texlive 2012
  %command=\hskip-2em,	%command
  %hang=,		%fit broad number
  before=\vskip.0em,		%command
  inbetween=\vskip.0em,		%command
  after=\vskip.0em,		%command
  %indentnext=,		%yes no
  indenting=yes,	%never not no yes always first next
]

% NOTE: you can only use the star/stop form of clglo
\def\startclglo{\dotripleargument\dostartclglo}
\long\def\dostartclglo[#1][#2][#3]#4\stopclglo{%
\sClGlo[#1]{#2}{#3}%
\reference[clglocn:#1]{#2}%
\startclglodsc{#2(#3)}%
#4
\stopclglodsc
}

\define[1]\cngloemp{{\ftEmp{\cnglo{#1}}}}
%%%%%%%%%%%%%%%%%%%%%%%%%%%%% reference %%%%%%%%%%%%%%%%%%%%%%%%%%%%%%%%%%%%%%%%

\if 0
\setupreferencing[
  %state	start stop
  %sectionnumber=yes,	%yes no
  %prefix	+ - text
  %interaction	label text all symbol
  %width	dimension
  left=,	%command	\about 默認會被“”包起來,此處移除雙引號
  right=,	%command
  %convertfile	yes no small big
  %separator	text
  %autofile	yes no page
  %global	yes no
]
\fi


\definereferenceformat[refclglo][
  left=,	%text
  right=,	%text
  %text=,	%text
  %label=,	%name
  command=\about,	%command % default is \in
]
\define[1]\refglo{%
\refclglo[clglocn:#1]%
}

\define[1]\refchapter{%
\in{第}{章}[chapter:#1]%
}

\define[1]\refappendix{%
\in{附錄}{}[appendix:#1]%
}

\define[1]\refappendixsec{%
\in{附錄}{}[sec:#1]%
}

\define[1]\refsec{%
\in{節}{}[sec:#1]%
}

\define[1]\refitem{%
\in{第}{項}[item:#1]%
}

\define[1]\reffig{%
\in{圖}{}[fig:#1]%
}
\define[1]\reftab{%
\in{表}{}[tab:#1]%
}
%%%%%%%%%%%%%%%%%%%%%%%%%%%%%%%%floats%%%%%%%%%%%%%%%%%%%%%%%%%%%%%%%%%%%%%%%%%
\setupframed[
  %height	fit broad dimension
  %width	fit broad fixed local dimension
  %autowidth	yes no force
  %offset	none overlay default dimension
  %location	depth hanging high lohi low top middle bottom
  %option	none empty
  %strut	yes no global local
  %align	no flushleft flushright middle normal high low lohi
  %bottom	command
  %top	command
  %frame	on off none overlay
  %topframe	on off
  %bottomframe	on off
  %leftframe	on off
  %rightframe	on off
  %frameoffset	dimension
  %framedepth	dimension
  %framecorner	round rectangular
  %frameradius	dimension
  %framecolor	name
  %background	screen color none foreground name
  %backgroundscreen	number
  %backgroundcolor	name
  %backgroundoffset	frame dimension
  %backgrounddepth	dimension
  %backgroundcorner	round rectangular
  %backgroundradius	dimension
  %depth	dimension
  %corner	round rectangular
  %radius	dimension
  %empty	yes no
  %foregroundcolor	name
  %foregroundstyle	name
  %rulethickness	dimension
]
\setupfloats[
  %location	left right middle
  %width	fit dimension
  before=,	%command
  after=,	%command
  %margin	dimension
  %spacebefore	small medium big none
  %spaceafter	small medium big none
  %sidespacebefore	small medium big none
  %sidespaceafter	small medium big none
  indentnext=yes,	%yes no
  %ntop	number
  %nbottom	number
  %nlines	number
  %default	name
  %tolerance	0 1 2
  %leftmargindistance	dimension
  %rightmargindistance	dimension
  %sidealign	normal line
  %numbering	yes nocheck
  % to see \setupframed
]

\setupcaptions[align={middle}]

\setuplabeltext[cn][figure={圖\;,}]
\setupcaption[figure][
  %location	top bottom none high low middle
  %width=max,	%fit broad max dimension
  %minwidth	fit dimension
  headstyle=\rmx\bf,	%normal bold slanted boldslanted type cap small... command
  style=\rmx\bf,	%normal bold slanted boldslanted type cap small... command
  %number	yes no
  %inbetween	command
  %align	left middle right no
  conversion=numbers,	%numbers characters Characters romannumerals Romannumerals
  way=bychapter,	%bytext bysection bychapter 每一章都重新编号
  %separator=a,	%text
  %stopper	text
  %command	command
  %distance	dimension
  prefixsegments=chapter,	%chapter:section
]

\setuplabeltext[cn][table={表\;,}]
\setupcaption[table][
  location=top,	%top bottom none high low middle
  %width=max,	%fit broad max dimension
  %minwidth	fit dimension
  headstyle={\rmx\bf},	%normal bold slanted boldslanted type cap small... command
  style={\rmx\bf},	%normal bold slanted boldslanted type cap small... command
  %number	yes no
  %inbetween	command
  %align	left middle right no
  conversion=numbers,	%numbers characters Characters romannumerals Romannumerals
  way=bychapter,	%bytext bysection bychapter 每一章都重新编号
  %separator=a,	%text
  %stopper	text
  %command	command
  %distance	dimension
  prefixsegments=chapter,	%chapter:section
]

\setupfloatsplitting[lines=1,conversion=greek]
%%%%%%%%%%%%%%%%%%%%%%%%%%%%%% formula %%%%%%%%%%%%%%%%%%%%%%%%%%%%%%%%%%%%%%%%%
\setupformulas[
  %location	left right
  %left	text
  %right	text
  %align	flushleft middle flushright
  %option	middle
  %strut	yes no
  %distance=0.01ex,	%dimension
  %margin	dimension standard yes no
  %leftmargin	dimension
  %rightmargin	dimension
  indentnext=auto,	%yes no
  %alternative	name
  %spacebefore=0.01ex,	%dimension
  %after	dimension
  %separator	text
  %conversion	numbers characters Characters romannumerals Romannumerals text
]

%%%%%%%%%%%%%%%%%%%%%%%%%%% matter independent %%%%%%%%%%%%%%%%%%%%%%%%%%%%%%%%%
%两个quad空格	a \qquad b
%quad空格	a \quad b
%大空格		a\ b
%中等空格	a\;b
%小空格		a\,b
%没有空格	ab
%紧贴		a\!b

\setuplabeltext[cn][part={第\;,\;卷}]
\setuphead[part][
  conversion=chinesecapnumeralscn,
  textstyle={\rm\bfd}, %normal bold slanted boldslanted % 標題內容
  numberstyle={\rm\bfd}, %normal bold slanted boldslanted % 編號
%  numbercommand=\PartNbr,
  header=high, %none empty high nomarking %章節首頁無頁眉
  footer=high, %none empty high nomarking
  before={\blank[2em]}, %COMMAND
  after={\blank[1em]}, %COMMAND
  alternative=middle, %normal inmargin middle TEXT
  placehead=yes,
]

\setuplabeltext[cn][chapter={第\;,\;章}]
% sectionblock中的設置無效,可能是bug, workaround is:
\setuplabeltext[cn][appendix={附錄\;,}]
\setuphead[chapter][
  conversion=numbers, %chinesenumerals, chinesecapnumeralscn chineseallnumeralscn
  sectionsegments=2:*,
  indentnext=yes,
  textstyle={\rm\bfc}, %normal bold slanted boldslanted % 標題內容
  numberstyle={\rm\bfc}, %normal bold slanted boldslanted % 編號
  number=yes, %yes no % 是否帶編號
  ownnumber=no, %yes no % 是否指定編號,如果是,則第一個參數就是編號
  page=yes, %left right yes % 是否固定於左頁或右頁
  continue=no, %yes no % 第一個是否緊接上一層,優先於*page*
  header=high, %none empty high nomarking %章節首頁無頁眉
%  text=nomarking, %none empty high nomarking
  footer=high, %none empty high nomarking
  before={\blank[2em]}, %COMMAND
%  inbetween=, %COMMAND
  after={\blank[1em]}, %COMMAND
  alternative=middle, %normal inmargin middle TEXT
  sectionnumber=no,
]

%\setuplabeltext[cn][section={節\;,}]
\setuphead[section][
  conversion=numbers, %chinesecapnumeralscn chineseallnumeralscn
  sectionsegments=2:*,
  indentnext=yes,
  textstyle={\rm\bfb}, %normal bold slanted boldslanted % 標題內容
  numberstyle={\rm\bfb}, %normal bold slanted boldslanted % 編號
  number=yes, %yes no % 是否帶編號
  ownnumber=no, %yes no % 是否指定編號,如果是,則第一個參數就是編號
  before={\blank[1.5em]}, %COMMAND
  after={\blank[1em]}, %COMMAND
  alternative=inmargin, %normal inmargin middle TEXT
]

\setuphead[subsection][
  conversion=numbers,
  indentnext=yes,
  textstyle={\rm\bfa}, %normal bold slanted boldslanted % 標題內容
  numberstyle={\rm\bfa}, %normal bold slanted boldslanted % 編號
  number=yes, %yes no % 是否帶編號
  ownnumber=no, %yes no % 是否指定編號,如果是,則第一個參數就是編號
  before={\blank[1em]}, %COMMAND
  after={\blank[1em]}, %COMMAND
  alternative=inmargin, %normal inmargin middle TEXT
]

\setuphead[subsubsection][
  conversion=numbers,
  indentnext=yes,
  textstyle={\rm\ita}, %normal bold slanted boldslanted % 標題內容
  numberstyle={\rm\ita}, %normal bold slanted boldslanted % 編號
  number=yes, %yes no % 是否帶編號
  ownnumber=no, %yes no % 是否指定編號,如果是,則第一個參數就是編號
  before={\blank[.5em]}, %COMMAND
  after={\blank[.5em]}, %COMMAND
  alternative=inmargin, %normal inmargin middle TEXT
]

\setuplist[
  alternative=c,	%a 空格
			%b 右对齐
			%c ...page
			%d 大空格 紧接下一个list
			%e 编号加方框,标题向左缩
			%f
			%g 标题居中... none command
  %label=yes,		% bug: 如果設置此項,則編譯參考文獻時會失敗
  interaction=all,
%  aligntitle=yes,
%  align=flushleft,
]
\setuplist[chapter][
  width=4em,	% setuplist中的設置無效,可能是bug
  margin=0em,	%dimension % 左侧缩进
  label=yes,
]

\setuplist[section][
  width=4em,
  margin=2em,	%dimension % 左侧缩进
  label=yes,
]

\setuplist[figure][
  width=4em,
  label=figure, % label=yes 無效,這可能是bug
  margin=0em,	%dimension % 左侧缩进
]

\setuplist[table][
  width=4em,
  label=table, % label=yes 無效,這可能是bug
  margin=0em,	%dimension % 左侧缩进
]

% page number
\definestructureconversionset[frontpart:pagenumber][][romannumerals]
\definestructureconversionset [bodypart:pagenumber][][numbers]
\definestructureconversionset [appendix:pagenumber][][numbers]
\definestructureconversionset [backpart:pagenumber][][numbers]
\setuppagenumber[numberconverionset=pagenumber]

\startsectionblockenvironment[frontpart]
\resetpagenumber

\setupfootertexts[text]
[]
[pagenumber]
[pagenumber]
[]

\setuphead[title][
  sectionsegments=2:*,
  indentnext=yes,
  textstyle={\rm\bfb}, %normal bold slanted boldslanted % 標題內容
  page=yes, %left right yes % 是否固定於左頁或右頁
  continue=no, %yes no % 第一個是否緊接上一層,優先於*page*
  header=high, %none empty high nomarking %章節首頁無頁眉
%  text=nomarking, %none empty high nomarking
  footer=normal, %none empty high nomarking
  before={\blank[2em]}, %COMMAND
%  inbetween=, %COMMAND
  after={\blank[1em]}, %COMMAND
  alternative=middle, %normal inmargin middle TEXT
  sectionnumber=no,
]

\stopsectionblockenvironment

\startsectionblockenvironment[bodypart]
\resetpagenumber
%\setupheader[state=none]
\define\BodyTextRightHeader{%
\framedtext[frame=off,bottomframe=on,width=broad,offset=none,frameoffset=2pt,]{%
\rlap{} \hfill {第\;\headnumber[chapter]\;章\;\;\getmarking[chapter]}\hfill \llap{}%
}%
}
% 由於可能某頁中沒有任何 section,所以不打印節編號,只打印節名。 no workaround
\define\BodyTextLeftHeader{%
\framedtext[frame=off,bottomframe=on,width=broad,offset=none,frameoffset=2pt,]{%
\rlap{} \hfill {\getmarking[section][first]}\hfill \llap{}%
}%
}

\setupheader[text][before={\vskip-4pt},]

\setupheadertexts[text]
[]
[\BodyTextRightHeader]
[]
[\BodyTextLeftHeader]

\setupfootertexts[text]
[]
[\centerline{第\;\pagenumber\;頁}]
[]
[\centerline{第\;\pagenumber\;頁}]

\define\BodyEdgeLeftFooter{\rotate[%
  rotation=-90,
  width=fit,
  height=fit,
  frame=off,
  offset=2pt,
  background=color,
  backgroundcolor=darkgray,
  foregroundcolor=white,
  corner=round,
]{熔接机软件设计说明书 —— \from[authorEmail]}}
\define\BodyEdgeRightFooter{\rotate[%
  rotation=90,
  width=fit,
  height=fit,
  frame=off,
  offset=2pt,
  background=color,
  backgroundcolor=darkgray,
  foregroundcolor=white,
  corner=round,
]{熔接机软件设计说明书 —— \from[authorBlog]}}
\setupfootertexts[edge]
[]
[\BodyEdgeLeftFooter]
[\BodyEdgeRightFooter]
[]
\stopsectionblockenvironment

\startsectionblockenvironment[appendix]
\setuphead[chapter][
  conversion=Characters,
]


%\setuplabeltext[cn][section={附錄\;,}]
\setuphead[section][
  conversion=Characters,
]
\setuphead[subsection][
  conversion=Characters,
]

\setupfootertexts[text]
[]
[pagenumber]
[pagenumber]
[]
%\resetpagenumber
\stopsectionblockenvironment

\startsectionblockenvironment[backpart]
\resetpagenumber
\stopsectionblockenvironment

%%%%%%%%%%%%%%%%%%%%%%%%%%%%%% pdf %%%%%%%%%%%%%%%%%%%%%%%%%%%%%%%%%%%%%%%%%%%%%
\placebookmarks[
  chapter,section,subsection,subsubsection% put these in the pdf index
][
  all% open all by default
]

\placebookmarks[
  title%
][
  force=yes%
]

\setupinteraction[
  title={熔接机软件设计说明书},
  author={倪庆亮},
  subtitle={Khronos OpenCL Working Group},
  keyword={OpenCL 1.2}
]
%%%%%%%%%%%%%%%%%%%%%%%%%%%%%%% Bibliographies %%%%%%%%%%%%%%%%%%%%%%%%%%%%%%%%%
\setupbibtex[
  database={ref},
  sort=author,	% title author short no FILE
]

\define[1]\WithBrackets{[#1]~}
\setuppublications[
  alternative=cn,	% ams apa apa-de apa-fr aps num num-fr ssa
  numbering=yes,
  numbercommand=\WithBrackets,
  autohang=yes
]
\setupheadtext[cn][pubs={參考文獻}]
%%%%%%%%%%%%%%%%%%%%%%%%%%%%%% url %%%%%%%%%%%%%%%%%%%%%%%%%%%%%%%%%%%%%%%%%%%%%
\setupurl[
  style={\rm\it},
  color=blue,
]
% url
\useurl[authorBlog][https://niqingliang2003.wordpress.com]
\useurl[authorGithub][https://github.com/walkthetalk]
\useurl[authorEmail][mailto:niqingliang2003@tom.com][][niqingliang2003@tom.com]
%%%%%%%%%%%%%%%%%%%%%%%%%%%%%%%%%%%%% makeup %%%%%%%%%%%%%%%%%%%%%%%%%%%%%%%%%%%
% \setupmakeup[
%name
%][
%width	dimension
%height	dimension
%voffset	dimension
%hoffset	dimension
%page	left yes right
%commands	command
%doublesided	yes no empty
%headerstate	normal stop start empty none nomarking
%footerstate	normal stop start empty none nomarking
%textstate	normal stop start empty none nomarking
%topstate	stop start
%bottomstate	stop start
%pagestate	stop start
%color	name
%]

\definelayout[frontCover]
\definemakeup[frontCover][
  page=right,
  doublesided=no,
  headerstate=none,
  footerstate=none,
  pagestate=stop,
]
%%%%%%%%%%%%%%%%%%%%%%%%%%%%%% fig %%%%%%%%%%%%%%%%%%%%%%%%%%%%%%%%%%%%%%%%%%%%%
%\defineframed[mplabel][frame=off,align=middle]
\define[1]\mplabel{%
\rm\itx%
\framed[frame=off,align=middle]{#1}%
}

\startMPinclusions
input grafbase;

numeric randomFactor;

randomFactor := 0.05;

u := 4mm;
v := 3.5mm;

vardef pathBox(expr xs, ys) =
	randompath (0,randomFactor) (unitsquare shifted (-0.5,-0.5) xscaled xs yscaled ys)
enddef;

vardef pathArrowBase(expr l) =
	numeric bx, by;
	bx := 0.1u;
	by := 0.2v;

	numeric axillax, axillay;
	axillax := l-0.3u;
	axillay := 0.025v;

	numeric scx, scy;
	scx := axillax/2;
	scy := (by - axillay) * 0.2 + axillay;

	numeric tipx, tipy;
	tipx := l-0.5u;
	tipy := 0.2v;

	numeric acx, acy;
	acx := l-0.25u;
	acy := 0.1v;

	randompath (0,randomFactor) (0, by)..(bx,0)..(0,-by)--
	(0,-by)..(scx, -scy)..(axillax,-axillay)--
	(axillax,-axillay)--(tipx,-tipy)--
	(tipx,-tipy)..(acx,-acy)..(l,0)--
	(l,0)..(acx,acy)..(tipx,tipy)--
	(tipx,tipy)..(axillax,axillay)--
	(axillax,axillay)..(scx,scy)..(0,by)--cycle
enddef;

vardef pathArrow(expr sp,ep) =
	numeric l;
	l := abs (ep - sp);

	pathArrowBase(l) rotated angle(ep-sp) shifted sp
enddef;

vardef itemArrowBase(expr sp,ep,fillColor,alphaChannel) =
image(
	path tmpPath;
	tmpPath := pathArrow(sp,ep);
	draw tmpPath;
	fill tmpPath withcolor transparent(1, alphaChannel, fillColor);
)
enddef;

% read only
vardef itemArrowRO(expr sp,ep) =
image(
	draw itemArrowBase(sp, ep, green, 0.7);
)
enddef;

% read and write
vardef itemArrowRW(expr sp,ep) =
image(
	draw itemArrowBase(sp, ep, red, 0.7);
)
enddef;

vardef itemArrow(expr sp,ep) =
image(
	draw itemArrowBase(sp, ep, white, 0.7);
)
enddef;

vardef txtFrame(expr xs, ys, lbl) =
image(
	draw randompath (0,randomFactor) (unitsquare shifted (-0.5,-0.5) xscaled xs yscaled ys);
	label(lbl, (0,0));
)
enddef;

vardef txtFrameDashed(expr xs, ys, lbl) =
image(
	draw randompath (0,randomFactor) (unitsquare shifted (-0.5,-0.5) xscaled xs yscaled ys) dashed evenly scaled 1;
	label(lbl, (0,0));
)
enddef;

vardef blankFrame(expr xs, ys) =
image(
	draw randompath (0,randomFactor) (unitsquare shifted (-0.5,-0.5) xscaled xs yscaled ys);
	label(btex \mplabel{} etex, (0,0));
)
enddef;
\stopMPinclusions
\startuniqueMPgraphic{frontcoverBG}
    path p;
    hu := 1cm; w := OverlayWidth; h := OverlayHeight;
    roundcorner := 8pt; offset := 12pt;

    color shade;
    shade := (.6,.6,.6);

    for i = 0 step .02 until 1:
        fill unitsquare xyscaled (w-i*hu,h-i*hu) squeezed (8pt,6pt)
             shifted (offset+i*hu/2,-offset+i*hu/2)
             withcolor transparent (1,.1,shade);
    endfor;

    p := unitsquare xyscaled (w, h) shifted (offset/2, -offset/2) squeezed (8pt,6pt);
    fill p withcolor \MPcolor{gray};
    draw p withcolor (0.6,0.8,1);
\stopuniqueMPgraphic
\defineoverlay[frontcoverBG][\uniqueMPgraphic{frontcoverBG}]
%%%%%%%%%%%%%%%%%%%%%%%%%%%%%%% english %%%%%%%%%%%%%%%%%%%%%%%%%%%%%%%%%%%%%%%%
% 在 definestartstop 中使用 setupindenting 等會修改全局設定,因此改用 \def
\def\startEnglish{\dostartEnglish}

\long\def\dostartEnglish#1\stopEnglish{%
{\setscript[no]
\mainlanguage[en]
\setupindenting[no]%
\setupwhitespace[line]%
\setupinterlinespace[medium]%
#1}
}

\def\startEnglishWithoutWhiteSpace{\dostartEnglishWithoutWhiteSpace}

\long\def\dostartEnglishWithoutWhiteSpace#1\stopEnglishWithoutWhiteSpace{%
{\setscript[no]
\mainlanguage[en]
\setupindenting[no]%
\setupwhitespace[medium]%
\setupinterlinespace[medium]%
\blank[1ex]
#1%
\blank[1ex]}
}
%%%%%%%%%%%%%%%%%%%%%%%%%%%%%% misc %%%%%%%%%%%%%%%%%%%%%%%%%%%%%%%%%%%%%%%%%%%%
% chinese date
\startluacode
local function tochineseYear(n)
	local cap = {
		["0"] = "零",
		["1"] = "壹",
		["2"] = "贰",
		["3"] = "叁",
		["4"] = "肆",
		["5"] = "伍",
		["6"] = "陆",
		["7"] = "柒",
		["8"] = "捌",
		["9"] = "玖",
	}

	s, p = string.gsub(string.format("%d",n), "(.)", function(s) return cap[s] end)
	return s
end

function commands.chinese_date() -- wrong namespace
	local temp = os.date("*t")

	tex.sprint(tex.ctxcatcodes,tochineseYear(temp.year))
	tex.sprint(tex.ctxcatcodes,"年")
	commands.chinesecapnumerals(temp.month)
	tex.sprint(tex.ctxcatcodes,"月")
	commands.chinesecapnumerals(temp.day)
	tex.sprint(tex.ctxcatcodes,"日")
end
\stopluacode

\define\COMPILEDATE{%
\ctxlua{commands.chinese_date()}%
}
%\setuplanguage[cn][date={year,年,month,月,day,日}]

% the textbackgournd will extends to the page boundary,
% which include the footnote and not as expected
% workaround is:
\chardef\kindofpagetextareas\plusone

% the footnote can't process linebreak correctly
% workaround is:
\let\stdfootnote\footnote
\def\footnote#1{%
\stdfootnote{\setscript[hanzi]#1}%
}
% \postponenotes\footnote{...}

% 中文caption第一行字符中间有空格
% workaround is:
\setupcaptions[align={broad,middle}]

\define[1]\todo{#1}
%\define[1]\todo{\error}
\define[1]\problem{}
\stopenvironment

