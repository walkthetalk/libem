%
% author:	Ni Qingliang
% date:		2011-02-11
%
\startcomponent cpn-arch
\product wm-sw-arch

\chapter{软件架构}
\section{总体架构}
软件总体架构如\reffig{wm_sw_arch} 所示,
共分为两大部分,一是应用(图的上半部分),一是驱动(图的下半部分)。
应用主要实现的是 GUI 以及熔接机的主要业务逻辑,另外还包含存档、读档等功能;
而此处的驱动为广义的驱动,不光包含驱动本身,还包含供应用使用的 Agent。
就总体架构而言,就这两个层次,层次较少有利于减少性能损耗。
目前考虑以多线程的方式实现,但在具体实施时可能会考虑往多进程迁移的可能,
尽量减少业务代码对多进程、多线程的依赖。



\startreusableMPgraphic{wmSwArch}
%picture picMain;
%pair pairUR, pairLL, pairC;
%pen penCmm;
%color memColor;

%ahangle := 30;
%ahlength := .5v;
%penCmm := pencircle scaled 2;
%memColor := (1,0.93,0.98);

vardef itemApp =
	txtFrame(6u,4v, btex \mplabel{应用} etex)
enddef;

vardef itemView =
	txtFrame(6u,2v, btex \mplabel{视图} etex)
enddef;

vardef itemCtrl =
	txtFrame(4u,v, btex \mplabel{控制器} etex)
enddef;

vardef itemModel =
	txtFrame(6u,2v, btex \mplabel{模型} etex)
enddef;

vardef itemApp =
image(
	draw txtFrame(7u,11v,btex \mplabel{} etex);

	draw itemView shifted (0,4v);
	draw itemCtrl;
	draw itemModel shifted (0,-4v);

	draw itemArrow((0,0.5v), (0,3v)) shifted(-u,0);
	draw itemArrow((0,3v), (0,0.5v)) shifted(u,0);

	draw itemArrow((0,-3v), (0,-0.5v)) shifted(-u,0);
	draw itemArrow((0,-0.5v), (0,-3v)) shifted(u,0);
)
enddef;

vardef itemDRV(expr lblA, lblD) =
image(
	draw txtFrame(2.5u,6v,btex \mplabel{} etex);

	draw txtFrame(2u,v, lblA) shifted (0,2v);
	draw txtFrame(2u,v, lblD) shifted (0,-2v);

	draw itemArrow((0,1.5v), (0,-1.5v));
)
enddef;

draw itemApp shifted (0, 5v);

draw itemDRV(btex \mplabel{Agent1} etex, btex \mplabel{Driver1} etex) shifted(-4.5u, -5v);
draw itemDRV(btex \mplabel{Agent2} etex, btex \mplabel{Driver2} etex) shifted(-1.5u, -5v);
label(btex \mplabel{\cdots} etex,(0,0))                               shifted( 1.5u, -3v);
label(btex \mplabel{\cdots} etex,(0,0))                               shifted( 1.5u, -7v);
draw itemDRV(btex \mplabel{Agent3} etex, btex \mplabel{Driver3} etex) shifted( 4.5u, -5v);


draw itemArrow((-3u,0), (-4.5u,-2.5v));
draw itemArrow((3u,0), (4.5u,-2.5v));
draw itemArrow((-u,0), (-1.5u,-2.5v));

\stopreusableMPgraphic


\placefigure[here][fig:wm_sw_arch]
{熔接机软件总体架构}
{\reuseMPgraphic{wmSwArch}}

\reffig{wm_sw_arch}中的箭头所展示的主要是调用关系。

\section{应用部分}
应用部分主要采用 MVC 模式,
此模式主要目的是将业务逻辑和 UI 分离开,
以便开发工作的分工和组件的重用,同时也可以比较方便的更换 GUI 框架(如果需要的话)。

【视图】就是熔接机软件的 GUI,负责数据的呈现;
【模型】就是熔接机软件的业务逻辑,比如各参数之间的约束关系,所有业务数据也存储在其中;
【控制器】负责【视图】和【模型】之间的通信。

应用部分所采用的 GUI 框架为 QT(E),
从 QT4 开始,加入了对 MVC 的支持,此模式可以很好的与 QT 协同工作。
【模型】部分应尽量减少对 QT 的依赖。

\subsection{(业务)模型}
此部分是整个熔接机软件的核心,所有业务逻辑均在其内。
根据熔接机软件的特点,此部分逻辑视图如\reffig{wm_model_arch}所示。


\startreusableMPgraphic{wmModelArch}
vardef itemGlobalCfg =
	txtFrame(26u,v, btex \mplabel{全局配置数据} etex)
enddef;

vardef itemController =
	txtFrame(27u,2v, btex \mplabel{\itc 控制器} etex)
enddef;

vardef itemSvc(expr lbl) =
image(
	draw txtFrame(6u,7v, btex \mplabel{} etex);
	label.lrt(lbl, (-3u,3.5v));
	draw txtFrame(1.5u, v, btex \mplabel{SPC} etex) shifted (0.5u, 2v);
	draw txtFrame(1.5u, 2v, btex \mplabel{状态\\数据} etex) shifted (-1u, -1.5v);
	draw txtFrame(1.5u, 2v, btex \mplabel{配置\\数据} etex) shifted (2u, -1.5v);

	draw itemArrowRO((1u,1.5v), (1.5u,-0.5v));
	draw itemArrowRW((0u,1.5v), (-0.5u,-0.5v));

	% external arrow
	draw itemArrow((0.5u,6v), (0.5u,2.5v));
	draw itemArrowRW((2u,6v), (2u,-0.5v));
	draw itemArrowRO((-1u,6v), (-1u,-0.5v));

	draw itemArrowRO((0.5u,1.5v), (0.5u,-4.5v));
)
enddef;

% Model
draw blankFrame(27u,12v);

draw itemSvc(btex \mplabel{业\\务\\流\\1} etex) shifted (-8u, v);
draw itemSvc(btex \mplabel{业\\务\\流\\2} etex) shifted (-1u, v);
label(btex \mplabel{\cdots} etex, (0,0))       shifted (3.5u,v);
draw itemSvc(btex \mplabel{业\\务\\流\\N} etex) shifted (8u,v);

draw itemGlobalCfg shifted (0,-4v);
draw itemArrowRW((-12u, 7v), (-12u,-3.5v));

label.llft(btex \mplabel{\itc 模\\型} etex, (0,0)) shifted (13.5u,6v);

% Controller
draw itemController shifted (0,8v);

% legend
vardef itemLengend =
image(
	label.rt(btex \mplabel{图例:} etex, (0,0)) shifted(-13u,0);

	draw itemArrowRO((-10u, 0), (-9u,0));
	label.rt(btex \mplabel{只读} etex, (0,0)) shifted (-9u, 0);

	draw itemArrowRW((-6u, 0), (-5u,0));
	label.rt(btex \mplabel{读写} etex, (0,0)) shifted (-5u, 0);

	draw itemArrow((-2u, 0), (-1u,0));
	label.rt(btex \mplabel{控制} etex, (0,0)) shifted (-1u, 0);
)
enddef;

draw itemLengend shifted (0, -7v);

\stopreusableMPgraphic

\placefigure[here][fig:wm_model_arch]
{业务模型}
{\reuseMPgraphic{wmModelArch}}

熔接机软件的一个特点就是大量配置数据,
\reffig{wm_model_arch}中除了这一点还有另外一个要素,
就是【业务流】。
这两个要素分别以动态和静态的形式对熔接机软件进行了描述。

【业务流】可以认为是熔接机的主要功能,如熔接、加热等,
每个【业务流】都是一个动态的过程,
这个动态过程由【SPC】表示,即业务流控制器,
他控制着业务的整个流程,或者说熔接、加热等功能就是由【SPC】来执行的。
在其执行过程中,
通过调用设备驱动的【Agent】来访问硬件(参见\reffig{wm_sw_arch})。
另外【SPC】在执行过程中还需要访问【配置数据】
(可能包括部分或全部【全局配置数据】,视每种业务的具体情况而定),
这些配置数据即用户所配置的一些参数,用于控制整个流程。
另外执行过程还会产生一些状态数据
(某些【状态数据】的变化可能需要通知【控制器】,\reffig{wm_model_arch}中没有画出)。

【控制器】会读写【(全局)配置数据】,
但只能读取【状态数据】,不能对其进行写操作。
另外【控制器】还会控制【SPC】的启动和停止
(某些业务的启动和/或停止可能是自动的,无须【控制器】介入)。

\startnotepar
配置数据之间的约束在【配置数据】内部完成。
每个【SPC】启动后以独立线程的形式执行相应的业务流。
每个【SPC】启动后可能会锁定对部分或全部【配置数据】(可能包括【全局配置数据】)的修改。
\stopnotepar

\startnotepar
此处讨论的【业务流】仅限于产品的特有功能,
不包括一些通用功能(如电池管理、系统时间管理等),
这些通用功能由【视图】部分直接与 OS 交互来完成。
\stopnotepar

\section{驱动部分}
每个驱动分成 Agent 和 Driver 两部分,
在 GNU / Linux 上,
Driver 为内核模块(.ko),
Agent 为动态库(.so)。
就我们的设备而言,不会出现两个程序共享同一设备的情况,
为了方便开发调试,我们采用【用户态驱动】。
虽然如此,但不一定严格按照【用户态驱动】的模式进行开发,
考虑到中断,每个驱动还是需要有内核模块来处理;
因此此处对【用户态驱动】进行推广,
就是将业务逻辑尽量放在用户空间执行(对应于 Agent),
除非必要,才将其放在内核空间中执行(对应于 Driver,如中断)。
当然每个驱动也可以根据实际需求,自行决定哪些工作放在用户空间执行,哪些工作放在内核空间执行。

如此, Driver 仅提供访问硬件的途径,并对其基本功能作简单的封装,
甚至仅提供 mmap,即内存映射,使得用户空间可以直接访问硬件寄存器。
在实现驱动部分时,需要考虑在 PC 上模拟运行的情况(GNU / Linux 的引入为模拟提供了极大的便利),
Driver 中的内容越少,则越容易模拟,反之则会比较困难。
理想情况下,除了 Driver,
应用部分以及驱动中的 Agent 均可以在 PC 上进行调测(需要实现 Driver 的模拟版本)。

相对于 Driver, Agent 提供了对硬件功能更高层次的抽象。
对每种硬件,需要仔细分析需求及其功能,使得定义的接口既能很好的屏蔽硬件细节,
又能提供很好的灵活性(以适配其他功能类似,型号不同的硬件)。

\section{其他}

\subsection{关于测试}

考虑到【自动化测试】的便利,可以考虑引入【CUI】。
由于目前我们考虑采用多线程的方式,【CUI】与【GUI】不方便共存。
所以业务部分,即【模型】可以做成程序库,
而【CUI】则可取代\reffig{wm_sw_arch}中的【视图】部分
(可能也包括【控制器】的部分功能)。

\subsection{关于大数据量}
软件中可能存在大数据量的传输,如视频的显示,
可以为这种数据开个绿色通道,由【视图】直接与相关驱动交互来完成数据传输。

\stopcomponent
