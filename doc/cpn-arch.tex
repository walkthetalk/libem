%
% author:	Ni Qingliang
% date:		2011-02-11
%
\startcomponent cpn-arch
\product fusion-splicer-SD

\chapter{软件架构}
软件总体架构如\reffig{fs_sw_arch} 所示,
由此图可以看出,本软件采用分层设计,从上至下依次为:
\startigBase
\item 「应用层」:此部分主要负责与用户的交互和配置管理;
\item 「业务层」:此部分主要负责实现熔接机相关的业务,比如熔接、加热等。
\item 「设备层」:此部分主要是对驱动层(系统调用)的封装,并将其重新组织成逻辑器件供上层使用,
比如高压板是由两部分组成,一是 GPIO 控制开关,一是可调电阻,驱动是两部分,而设备层将其封装在一起。
\item 驱动层:此部分主要是屏蔽底层硬件细节,同种设备向上层提供统一的接口。
\stopigBase


\setupFLOWcharts[nx=6,ny=4,
  dx=.25\bodyfontsize,dy=.5\bodyfontsize,
  width=6\bodyfontsize, height=3\bodyfontsize,
  maxwidth=.8\textwidth]

\startFLOWchart[fs_sw_arch]

\includeFLOWchart[app][x=1,y=1]

\startFLOWcell
\name{app}\location{1,1}\shape{none}
\text{应用层}
\stopFLOWcell

\startFLOWcell
\name{gui}\location{2,1}\shape{16}
\text{视图管理}
\stopFLOWcell

\startFLOWcell
\name{cfg}\location{3,1}\shape{16}
\text{配置管理}
\stopFLOWcell

\startFLOWcell
\name{msg}\location{4,1}\shape{16}
\text{消息管理}
\stopFLOWcell

\startFLOWcell
\name{input}\location{5,1}\shape{16}
\text{输入设备管理}
\stopFLOWcell

\startFLOWcell
\name{otherApp}\location{6,1}\shape{16}
\text{。。。}
\stopFLOWcell

\startFLOWcell
\name{svc}\location{1,2}\shape{none}
\text{业务层}
\stopFLOWcell

\startFLOWcell
\name{fs}\location{2,2}\shape{}
\text{熔接业务}
\stopFLOWcell

\startFLOWcell
\name{heat}\location{3,2}\shape{}
\text{加热业务}
\stopFLOWcell

\startFLOWcell
\name{cal}\location{4,2}\shape{}
\text{放电校正}
\stopFLOWcell

\startFLOWcell
\name{motortest}\location{5,2}\shape{}
\text{电机测试}
\stopFLOWcell

\startFLOWcell
\name{other}\location{6,2}\shape{}
\text{。。。}
\stopFLOWcell

\startFLOWcell
\name{dev}\location{1,3}\shape{none}
\text{设备层}
\stopFLOWcell

\startFLOWcell
\name{camera}\location{2,3}\shape{18}
\text{摄像头}
\stopFLOWcell

\startFLOWcell
\name{motor}\location{3,3}\shape{18}
\text{电机}
\stopFLOWcell

\startFLOWcell
\name{heater}\location{4,3}\shape{18}
\text{加热器}
\stopFLOWcell

\startFLOWcell
\name{sensor}\location{5,3}\shape{18}
\text{各种传感器}
\stopFLOWcell

\startFLOWcell
\name{otherSensor}\location{6,3}\shape{18}
\text{。。。}
\stopFLOWcell

\startFLOWcell
\name{drv}\location{1,4}\shape{none}
\text{驱动层}
\stopFLOWcell

\startFLOWcell
\name{camif}\location{2,4}\shape{18}
\text{camIF}
\stopFLOWcell

\startFLOWcell
\name{fpga}\location{3,4}\shape{18}
\text{FPGA}
\stopFLOWcell

\startFLOWcell
\name{gpio}\location{4,4}\shape{18}
\text{GPIO}
\stopFLOWcell

\startFLOWcell
\name{iic}\location{5,4}\shape{18}
\text{I2C 设备}
\stopFLOWcell

\startFLOWcell
\name{otherDrv}\location{6,4}\shape{18}
\text{。。。}
\stopFLOWcell

\stopFLOWchart

\placefigure[force][fig:fs_sw_arch]{熔接机软件总体架构}
            {\FLOWchart[fs_sw_arch]}

\section{层次之间的交互}
在这四层架构中,只有「驱动层」运行在内核空间,其他三层都运行在用户空间。

「应用层」与「业务层」之间主要通过消息进行交互(实现时采用 socket)。
「业务层」与「设备层」之间通过 API 调用进行交互。
「设备层」与「驱动层」之间通过系统调用进行交互。
另外在使用输入设备时,如按键、鼠标等,「应用层」也会直接与「设备层」交互。

详细交互流程请参考《熔接机软件设计说明书——应用部分》。

\section{SOA 和 OO }
整个软件采用面向服务的体系架构(SOA),此架构的一个特点就是松耦合,
服务间的接口是中立的,我们的系统中,服务间都是通过 socket 消息进行交互,即消息是中立的,
不依赖于任何服务,不依赖于特定操作系统,也不依赖于特定编程语言。

「业务层」的每一项业务都提供的是一种服务,其实「应用层」也可以认为提供的是一项服务,
这样理解的话,「应用层」其实是「业务层」的一部分。

具体到单项服务本身,我们采用面向对象(OO)的方式实现,这在后续章节中会有所体现。

接下来的章节将介绍「业务层」的实现。

\stopcomponent
