%
% author:	Ni Qingliang
% date:		2011-02-11
%
\startcomponent cpn-overview
\product fusion-splicer-SD

\chapter{技术选择}
相对于现有产品,下一代熔接机软件的最大变化在于 OS 的引入,
OS 的引入屏蔽了许多硬件细节,
使得我们可以更加专注于业务。
进而使得我们可以基于OS构建我们自己的软件平台,
以支持多种产品,并减少代码复用的难度;
在形成一个完整的软件平台后,可以缩短产品的开发周期,有利于快速推出新产品。
目前我们计划选用 GNU / Linux,
这也是目前的趋势,在嵌入式软件领域,还有其他许多 OS,
如 VxWorks、 Windows Embedded 等,
 GNU / Linux 最大的好处在于免费,
而且由于社区大量开发人员的参与,
很容易找到交流的对象,芯片厂商的支持目前也已经很成熟了。
另外,我们还可以方便的重用其他开源软件,
以减少我们的工作量。

\useURL[yoctoAddr][https://www.yoctoproject.org/]
\startnotepar
严格意义上讲, GNU / Linux 只是 OS 的内核,
诸如 Arch、 openSUSE、 Redhat 才能称为 OS,
此处我们选用 GNU / Linux 作为 OS 内核,
用 Yocto (是一个 OS 构建系统,基于 OpenEmbedded,详情参见\from[yoctoAddr])
来构建我们自己的 OS,或者叫 Linux Distribution。
\stopnotepar

GUI 是熔接机软件的一个重要组成部分,
为了照顾老客户,应当尽量减少其变动;
而由于触摸屏的引入,可能会有所调整,
但要遵循 PLA (principle of lease astonishment)。

对于 GUI,目前考虑还是继续采用 QT(E),
当然也有其他 GUI 框架,如 Android,
但是 Android 的性能需要进一步评估
(Google 发布 NDK 应该也有性能方面的原因),
目前我们对 Android 不是太熟悉,
保守起见,我们还是选用 QT(E)。

\useURL[sailFishAddr][https://sailfishos.org/]
\startnotepar
本来 GUI 是 OS 的一部分,此处我们选用 QT 只用在我们自己的应用程序上, OS 没有囊括 GUI。
当然,现在也有基于 QT 的嵌入式 OS,比如 SailFish(参见 \from[sailFishAddr]),
后续开发可以考虑。

另外 QT 本身是跨 OS 的,如果我们的其他产品用了 Windows Embedded 等 OS,
可以保证我们的 GUI 在不同产品上的一致性。
\stopnotepar

至于开发语言,嵌入式领域用的最多的就是 C 和 C++,
另外 QT 本身的开发语言就是 C++,
所以我们考虑采用 C++,
虽然 C++ 易学难精,但是一般的开发应该不会有太大问题,
选用 C++ 主要是要利用其面向对象的特性,
面向对象也是目前软件开发中最常用的编程思想,
虽然用 C 也可以写出面向对象的代码,
但毕竟不如语言本身就支持此特性来的方便,
当然面向对象还具备诸如易维护、质量高、效率高、易扩展等其他好处,
也会为我们的软件开发工作带来极大便利。

\startnotepar
虽然 Android 采用 Java 为主要编程语言,
但是 NDK 的引入使其可以与 C / C++ 开发的代码方便得进行交互。
如果以后考虑采用 Android,
那么可以仅用 Java 实现 GUI,
而业务逻辑还是采用 C / C++ 来开发。
\stopnotepar

熔接机软件还有一个特点,就是大量的参数。
在架构设计时也要充分考虑。
充分利用 C++ 的面向对象特性,
将所有参数分门别类,进行层次划分,以达到“高内聚”与“低耦合”的目标。

\section{关于开发过程的控制}

引入软件版本控制,
软件的版本控制最主要的是管理源代码。
目前业界已经都转向了分布式版本控制系统,如 git、 mercurial。
两种工具就功能而言没有太大区别,使用那种完全取决于个人喜好。
目前 Linux 内核、各种嵌入式 OS 构建系统等使用的都是 git,
我们在开发时不可避免要使用这些软件,
为了更好的互操作,选用 git 作为我们软件的版本控制系统。

\startnotepar
dailybuild、 autotest 等可以在开发过程种视具体情况逐步实施。
\stopnotepar

\stopcomponent

