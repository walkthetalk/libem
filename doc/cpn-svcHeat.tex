\startcomponent cpn-svcHeat
\product fusion-splicer-SD

\chapter{业务:加热}

加热是熔接机的另一项核心业务,本节主要介绍此业务的实现。

\section{消息}
加热业务主要与 UI 部分进行交互,所使用的消息包括:
\startigBase
\item heat_start:来自 UI,用于启动加热;
\item heat_stop:来自 UI,用于停止加热;
\item heat_state:发往 UI,用于报告状态。
\stopigBase
目前实现中业务所需的配置数据,如加热时间、加热温度等,均来由 fs_start 消息,
将来可以添加一个不带配置数据的 start 消息,这样只有第一次启动加热的时候需要发送 heat_start,
后面如果配置数据没有变化,就只需发送不带配置数据的 start 消息即可。

\section{流程}

\reffig{heat_sm}是加热业务的主要流程图。

\placefigure[here][fig:heat_sm]
{加热业务流程图}
{\externalfigure[mp/heat_process-1.pdf]}

由\reffig{heat_sm}可以看出,加热业务主要有两大状态,一为「idle」,即空闲态;
一为「running」,即运行态。

「idle」是加热业务的初始状态。目前此状态下主要处理的事件是 evStart,
由此业务的控制器根据消息 heat_start 转换而来。
当收到此事件后,转变为「running」状态。

「running」状态中主要由三种状态:
\startigBase[indentnext=no]
\item ascending:温度上升态;
\item descending:温度下降态;
\item cooling:冷却态。
\stopigBase
对于以上三种状态的任一种而言,
如果收到了 evStop 事件(由控制器根据消息 heat_stop 转换而来),
或者出现了错误则跳转到「idle」态,
如果是出现了错误,还要通过控制器发送消息 heat_state 上报错误讯息。
加热结束也会跳转到「idle」态,并上报状态。
此状态下所处理的事件列表:
\startigBase
\item evStop:停止加热,收到此事件后会转入「idle」态,并执行一些初始化操作,比如停止定时器。
\item evEntryAct:此事件为状态机内部事件,由状态机进入某一个状态时自发自收,
比如,进入「cooling」状态后就会发送此事件,接收到此事件后会检查温度是否已经低于预设值,
如果是,则「running」状态结束,并跳转到「idle」。
\item evHeatTimeout:加热时间已经达到了预设值,需要停止加热。
\item evCheckTimeout:周期性定时消息,用于检查温度。
\stopigBase

\startnotepar
控制器在接收到配置数据后需要判断是否位于「running」态,如果是,则丢弃配置数据并回告错误讯息。
即加热业务运行时不允许修改配置数据。

另外,「running」中的子状态发生变化时(正常流程)是否需要上报?
如果 UI 上除图像外还要以文字的形式显示当前状态,则需要上报,暂时不上报,如有需要再添加。
\stopnotepar

\section{类图}

\reffig{heat_class_diag} 是加热业务的类图(仅包括主要类)。
\placefigure[here][fig:heat_class_diag]
{加热业务类图}
{\externalfigure[mp/heat_class_diag-1.pdf]}
由图中可以看出,加热业务主要由三个类构成:
\startigBase
\item svcHeating:是此业务的核心类,继承自执行模型中的「poller」,
相当于业务的「控制器」;
\item heat_sm:是此业务的「状态机」;
\item heat_ctx:是此业务的「上下文」。
\stopigBase
下面分别介绍这三个类(只由代码主干,其他以 \ccmm{...} 取代)。

\subsection{类: svcHeating}

\startCPP
class svcHeating : public exemodel::poller {
	...
private:
	/// 处理 poller 由 m_svr 收到的消息
	void __svr_cb(exemodel::serveree::args_t & args);
	/// 用于检查加热器温度的周期性定时器回调
	__checkTmr_cb(timeree::args_t & args);
	/// 用于停止加热的一次性定时器回调
	__heatTmr_cb(timeree::args_t & args);

	/// 状态机用此函数上报状态
	__report(const std::string &);
private:
	exemodel::serveree m_svr;	/// server,用于与其他模块通信
	msg::recver m_recver;		/// 消息接收器,与 m_svr 配合使用
	msg::sender m_sender;		/// 消息发送器,与 m_svr 配合使用

	timeree m_check_tmr;		/// 用于检查加热器温度的周期性定时器
	timeree m_heat_tmr;		/// 用于停止加热的一次性定时器

	heater m_heater;		/// 加热器
	temp_sensor m_tmp_sensor;	/// 温度传感器

	cfg_data_t m_cfg;		/// 配置数据

	heat_ctx m_ctx;			/// 上下文
	heat_sm m_sm;			/// 状态机
};
\stopCPP

此类在构造函数中会将加热器、温度传感器、配置数据等的引用添加到 \cvar{m_ctx} 中,
将 \cvar{__report} 和 \cvar{m_ctx} 都传递给 \cvar{m_sm}。

\capi{__svr_cb} 中会处理收到的消息,并将其转换成事件发送给 \cvar{m_sm}。

\subsection{类: heat_ctx}

\startCPP
class heat_ctx {
	...
public:
	timeree & m_check_tmr;		/// 用于检查加热器温度的周期性定时器
	timeree & m_heat_tmr;		/// 用于停止加热的一次性定时器

	heater & m_heater;		/// 加热器
	temp_sensor & m_tmp_sensor;	/// 温度传感器

	const cfg_data_t & m_cfg;		/// 配置数据
};
\stopCPP

此类中主要存储的是相关设备以及配置数据的引用,注意配置数据时 \cqlf{const} 引用,
即状态机在使用此类时不能修改配置数据。为了方便状态机的使用,本来可以为此类添加一些接口,
但考虑到控制的复杂性,可能要添加的接口很多,因此所有引用都做成 \ckey{public} 的,
由状态机直接访问。

\subsection{类: heat_sm}

\startCPP
struct stIdle;	/// 初始状态
struct heat_sm : boost::statechart::state_machine< heat_sm, stIdle > {
	...
public:
	heat_ctx & m_ctx;	/// 上下文的引用
	std::function<void(const std::string & )> m_reporter;	/// 用于上报状态
};
\stopCPP

状态机里面主要由两项内容:一是上下文的引用,一是用于上报状态的回调,均在构造时确定。

\reftab{heat_state_list}中列出了加热业务状态机中的所有状态及相应的类名。
\reftab{heat_evt_list}中列出了加热业务状态机所处理的所有事件及相应的类名。

\startcolumns
\placetable[here][tab:heat_state_list]{加热业务状态列表}{

\bTABLE
\setupTABLE[c][1][align={flushright,lohi}]
\setupTABLE[c][2][align={flushleft,lohi}]

\bTABLEhead
\bTR
\bTD 类名 \eTD\bTD 简介 \eTD
\eTR
\eTABLEhead

\bTABLEbody

\bTR
\bTD \ctype{stIdle} \eTD\bTD 空闲态 \eTD
\eTR
\bTR
\bTD \ctype{stRunning} \eTD\bTD 运行态 \eTD
\eTR
\bTR
\bTD \ctype{stAscending} \eTD\bTD 温度上升态 \eTD
\eTR
\bTR
\bTD \ctype{stDescending} \eTD\bTD 温度下降态 \eTD
\eTR
\bTR
\bTD \ctype{stCooling} \eTD\bTD 冷却态 \eTD
\eTR

\eTABLEbody

\eTABLE

}
\placetable[here][tab:heat_evt_list]{加热业务事件列表}{

\bTABLE
\setupTABLE[c][1][align={flushright,lohi}]
\setupTABLE[c][2][align={flushleft,lohi}]

\bTABLEhead
\bTR
\bTD 类名 \eTD\bTD 简介 \eTD
\eTR
\eTABLEhead

\bTABLEbody

\bTR
\bTD \ctype{evStart} \eTD\bTD 启动加热 \eTD
\eTR
\bTR
\bTD \ctype{evStop} \eTD\bTD 停止加热 \eTD
\eTR
\bTR
\bTD \ctype{evEntryAct} \eTD\bTD 状态入口事件 \eTD
\eTR
\bTR
\bTD \ctype{evCheckTimeout} \eTD\bTD 检查温度定时器超时事件 \eTD
\eTR
\bTR
\bTD \ctype{evHeatTimeout} \eTD\bTD 加热超时事件 \eTD
\eTR
\eTR

\eTABLEbody

\eTABLE

}
\stopcolumns

下面是加热业务几个主要状态的具体实现:

\startCPP
/*
 * state: running
 */
struct stAscending;
struct stRunning : sc::simple_state< stRunning, heat_sm, stAscending > {
	typedef sc::transition< evStop, stIdle > reactions;
public:
	stRunning(void)
	{
		/// \todo 进入此状态时启动两个定时器
	}

	~stRunning()
	{
		/// \todo 退出此状态时停止两个定时器
	}
};

/*
 * state: ascending
 */
struct stDescending;
struct stCooling;
struct stAscending : sc::state< stAscending, stRunning > {
	typedef boost::mpl::list<
		sc::transition< evHeatTimeout, stCooling >,
		sc::custom_reaction< evCheckTimeout >
	> reactions;
public:
	stAscending(my_context ctx)
	: my_base(ctx)
	{
		/// \todo 进入此状态时打开加热器
	}

	~stAscending()
	{
		/// \todo 退出此状态时关闭加热器
	}

	sc::result react(const evCheckTimeout &)
	{
		if (温度超过预设值) {
			return transit< stDescending >();
		}

		/// \todo 如果温度接近预设值可能需要调整加热策略,
		/// 比如关闭发热量大的,打开发热量小的。

		return discard_event();
	}
};

/*
 * state: descending
 */
struct stDescending : sc::state< stDescending, stRunning > {
	typedef boost::mpl::list<
		sc::transition< evHeatTimeout, stCooling >,
		sc::custom_reaction< evCheckTimeout >
	> reactions;
public:
	stDescending(my_context ctx)
	: my_base(ctx)
	{
	}

	sc::result react(const evCheckTimeout &)
	{
		if (温度低于预设值) {
			return transit< stAscending >();
		}

		return discard_event();
	}
};

/*
 * state: cooling
 */
struct stCooling : sc::simple_state< stCooling, stRunning > {
	typedef sc::custom_reaction< evCheckTimeout > reactions;
public:
	sc::result react(const evCheckTimeout &)
	{
		/// \todo check sensor tmperature
		return transit< stIdle >();
	}
};
\stopCPP

\stopcomponent
